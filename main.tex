\documentclass[float=true]{ctexart}

\title{未名的北大游戏}
\author{Lunatic Works}
\date{\today}

\usepackage{advdate}    % Advancing/saving dates
\usepackage[calc,showdow,english]{datetime2}
\DTMnewdatestyle{mydateformat}{%
  \renewcommand{\DTMdisplaydate}[4]{%
    % \DTMshortmonthname{##2}\nobreakspace%  (full) Month
     ##1/##2/%  (full) Month
    ##3,\space%                day,
    \DTMshortweekdayname{##4}% short weekday,
    % \number##1%                       year
  }%
  \renewcommand{\DTMDisplaydate}{\DTMdisplaydate}%
}
\DTMsetdatestyle{mydateformat}
\usepackage{subfiles}
\usepackage{hyperref}
\usepackage{longtable}
\usepackage{xcolor}
\usepackage{multirow}
\hypersetup{
    colorlinks=true,
    linkcolor=blue,
    filecolor=magenta,      
    urlcolor=cyan,
}
\newcommand{\mydate}{2019-2-18}
\usepackage{xstring}

\definecolor{DOrange}{RGB}{145,20,20}
\definecolor{Red}{RGB}{255,0,0}
\definecolor{DeepGreen}{RGB}{5,102,8}
\newcommand\ZZ[1]{\textcolor{purple}{\texttt{ZZ: #1}}}
\newcommand\akr[1]{\textcolor{DOrange}{\texttt{Akarin: #1}}}
\newcommand\Ham[1]{\textcolor{DeepGreen}{\texttt{Hamster: #1}}}

% timeline
\newcommand{\timeline}[2]{\date{\DTMdate{2019-2-18 + #1 + #2}}}
\newcommand{\ruxuedate}{0}

\begin{document}

\maketitle

\newpage
% macro may help here
\newcommand{\cangshua}{{事件开始:风纪委员在某次关门收拾的时候发现少了一把椅子。恰好男主在活动室整理资料出来晚了,于是被指为犯人。男主辩称自己是无辜的,被要求说如果你是无辜的就自己找出犯人,男主无奈开始查案。
解决:男主一番查案无果,怀疑是灵异所为,但风纪委员不相信灵异。男主一天对风纪委员说你看锁门之后也是会丢的,不信我们来试一试,风纪委员应约,发现不仅东西丢了,还遇到许多其他吓人现象,被吓的不轻,也只好相信。(附加设定:之后会怕鬼)
幕后:男主在查案无果后只好选择向主角团求助,众人合伙上演了一波灵异事件。风纪委员所见其实是其他人所为。(系统:可以增加一些躲避小游戏?)最后其他人被关在新太阳里面一晚上,在活动室彻夜聊天。
结尾:众人问道椅子究竟去哪儿了。男主直言不知道。众人调侃着一定是你弄坏了,记得买把新的放回去一类,男主说真的不是我。事件结束,风机委员一边担心着闹鬼一边锁门。在门背后阴影中,似乎有什么东西拖着椅子消失在黑暗中……}}

\newcommand{\nvzhumeng}[1]{
    \IfEqCase{#1}{
        {1}{
            【男主角】遇到了一只可爱的猫猫。猫猫非常聪明听话,让它做什么它就做什么.渐渐地主角开始犯懒,让猫猫帮着打扫房间(可选选项),做家务,没想到猫猫都可以胜任。
        }
        {2}{
            渐渐地主角开始犯懒,让猫猫帮着打扫房间(可选选项),做家务,没想到猫猫都可以胜任。猫猫也和主角团熟悉起来,女主最喜欢猫猫了,因为它和女主过去养的一只猫很像。
        }
        {3}{
            与此同时,男主角和女主角参加【民俗学】课程,课程的大作业是探寻一个本地的民俗传说。主角选择了【传说A:据说是可以吞噬人灵魂的石像】,而女主角选择了【传说B】。男主角认为自己在专业课上比不过北大学霸,但是在民俗学这门“不务正业的课”上面总要证明自己,所以他提出两个人都先收集资料,看谁收集的资料全,那么就采用谁的主题。
        }
        {4}{
            在收集资料的途中,猫猫帮助了男主角很多,男主角了解到【传说A】实际上是起源于四十年前,那时候有一个北大宿舍里的所有学生都突然昏迷不醒。他们已经住院四十年了,同时流传出来他们的灵魂被吃掉了。这件事之后,没几年北大都会有昏迷不醒的学生出现。
        }
        {5}{
            主角去了医院,在医院找到了一个昏迷不醒的人和他们的家属,像他们了解情况之后,家属给了他一些当事人的所有物。他们似乎是前往山里探险,主角还拿到了十几年前他们拍摄的【底片】。            主角将底片洗出来,发现了那是一个山洞之中的石像,看上去不像是假的。得到了决定性证据的主角十分高兴,他在资料方面收集得更全,因而也让女主角同意最后大作业写【传说A】。
        }
        {6}{
            猫猫和女女主角一切准备大作业。女主角在进行PPT制作的同时,男主角感觉身体越来越不好,白天总是处于半梦半醒的状态,晚上则总是梦见底片里的那个山洞。
        }
        {7}{
            猫猫和女女主角一切准备大作业。女主角在进行PPT制作的同时,男主角感觉身体越来越不好,白天总是处于半梦半醒的状态,晚上则总是梦见底片里的那个山洞。
        }
        {8}{
            他开始逐渐怀疑自己是不是也被吞噬了灵魂。但是这个时候他已经卧病在床,女主角担心打电话过来,他甚至没有办法接电话。这时候猫猫跑过来看了他一眼,那个眼神让主角记忆深刻,然后主角就陷入了梦境。他一直徘徊在那个山洞之中,原来他被封印在了山洞中的石像之中。过去几十年被封印的灵魂们也都在,他们告诉主角,任何直接或者间接看到石像的人,最后都会被石像收走灵魂。这里的人等了几十年,也看不到任何被解救的希望,男主角听完后陷入了绝望。
        }
        {9}{
            没想到不久之后,猫猫竟然找到了山洞。主角大声对着猫猫疾呼,让猫猫来救自己。猫猫似乎听懂了主角的话,尝试着打碎石像。没想到每次攻击石像,石像身上都发出一道【电光】,将猫猫电得直叫。男主角处于灵魂的视角,他可以看到猫猫虽然身体受伤不大,但是灵魂上似乎变得萎靡。相对的,石像也似乎受到了一些伤害。男主角和被封印的人非常激动,他们看到了被解救的希望,因而指挥猫猫继续努力。随着猫猫变得越来越虚弱,男主角开始觉得让猫猫这样牺牲救自己不太好,有些愧疚。他想要阻止猫猫,但是周围的人却不断地要求猫猫继续努力,男主角不好开口说出阻止的话(有组织的选项,但是是灰色的选不了),毕竟这里有这么多人等到救援。再说了,女主角也看过【底片】,说不定过段时间也会被封印进来,男主角不想女主角也落到和自己一个下场。他用这个说法说服了自己。最后,猫猫终于打碎了石像,而主角也回到了自己的身体之中。
        }
        {10}{
            猫猫虽然也在不久之后回来了,但是却变得痴痴呆呆的,也不吃东西,逐渐消瘦下去。女主角虽然不知道发生了什么事,但是还是感到非常伤心,甚至表示希望想尽一切办法救助猫猫。
        }
        {11}{
            主角开始反思自己,当时他为了让猫猫救自己,甚至用女主角来当借口。主角感到十分愧疚,他想一切要是可以重来多好,他不想再坚持【传说A】了,这样一切悲剧都不会发生。
        }
    }[\PackageError{tree}{Undefined option to tree: #1}{}]%
}

\newcommand{\nvzhu}[1]{
    \IfEqCase{#1}{
        {0}{
            男主角和女主角选课:【民俗学】。
        }
        {1}{
            主角醒来,发现之前的一切似乎是一个梦,而自己又重新回到了 \DTMdate{\mydate+ 14},只是他不再是人而是那只猫猫。【原身】遇到了一只可爱的猫猫(主角)。猫猫非常聪明听话,让它做什么它就做什么.渐渐地原身开始犯懒,让猫猫帮着打扫房间(可选选项),做家务(复刻一周目选项,主角恨得牙痒痒但是都可以胜任了。
        }
        {2}{
            渐渐地原身开始犯懒,让猫猫帮着打扫房间(可选选项),做家务(复刻一周目选项,主角恨得牙痒痒但是都可以胜任了。主角到处探索,适应身体,并且偷偷开始调查传说。
        }
        {3}{
            与此同时,原身和女主角参加【民俗学】课程,课程的大作业是探寻一个本地的民俗传说。原身选择了【传说A:据说是可以吞噬人灵魂的石像】,而女主角选择了【传说B】。原身认为自己在专业课上比不过北大学霸,但是在民俗学这门“不务正业的课”上面总要证明自己,所以他提出两个人都先收集资料,看谁收集的资料全,那么就采用谁的主题。变成猫猫的主角想要阻止原身选择【传说A】,但是原身非常固执(复刻第一周目选项)。
        }
        {4}{
            在收集资料的途中,男主角了解到【传说A】实际上是起源于四十年前,那时候有一个北大宿舍里的所有学生都突然昏迷不醒。他们已经住院四十年了,同时流传出来他们的灵魂被吃掉了。这件事之后,没几年北大都会有昏迷不醒的学生出现。主角意识到,对于原身来说“放弃【传说A】”很简单,但是对于不能说话的猫猫来说却千难万难。原身最后在医院得到了【底片】,他踌躇满志,认为这次大作业肯定要选自己的题目了。
        }
        {5}{
            原身去了医院,在医院找到了一个昏迷不醒的人和他们的家属,像他们了解情况之后,家属给了他一些当事人的所有物。他们似乎是前往山里探险,原身还拿到了十几年前他们拍摄的【底片】。            原身将底片洗出来,发现了那是一个山洞之中的石像,看上去不像是假的。得到了决定性证据的原身十分高兴,他在资料方面收集得更全,他踌躇满志,认为这次大作业肯定要选自己的题目了。主角意识到,对于原身来说“放弃【传说A】”很简单,但是对于不能说话的猫猫来说却千难万难。主角猫猫偷偷将底片毁掉。
        }
        {6}{
            原身在和女主角对资料的时候,发现底片竟然已经被猫猫毁了,气得要命想要惩罚猫猫,但是被女主角拦下。失落的原身不想在大作业划水,他认为女主角收集了资料,那么就应该由自己来做PPT。女主角意识到男主角的低落,是不是来看望他。主角猫猫觉得阻止了原身作死大功告成的时候,却意外地注意到女主角离开时的眼神十分熟悉,就像是一周目猫猫的眼神一样。
        }
        {7}{
            主角猫猫越来越不安,他发现女主角神神秘秘地不知道在干什么,于是主角猫猫混在女主角身边,两个人逐渐熟悉。
        }
        {8}{
            有一天主角猫猫趁女主角外出的时候来到她的卧室,发现她竟然在整理【传说A】的资料。主角猫猫通过这些天对女主角的了解,明白了她的想法。她一定是不想看着男主角的努力白费,因此想要重新拍一张山洞里的石像的照片,最后大作业还是选择【传说A】。主角猫猫想,为什么这个女人这么傻,并且它意识到女主角可能已经动身前往山里了。女主角的桌子上标注了她认为可能的几个地点,主角猫猫根据一周目的记忆,判断哪一个最有可能(小游戏),并且它体会到一周目猫猫能找到自己有多不容易。
        }
        {9}{
            主角猫猫选择到了正确的地点,发现女主角已经倒在了石像旁边,她的灵魂正在被石像吞噬。主角猫猫选择和一周目猫猫一样,试着打碎石像,而石像每次反击的【闪电】都让他痛彻心扉。被封印的灵魂们不断地指挥主角猫猫,却不体谅它的痛苦,让主角猫猫十分不爽,但是他还是想要赶紧打碎石像,拯救女主角。这时候,女主角却勉强站了起来,帮助主角猫猫挡下了闪电。她要和主角猫猫一切承担,不想看到主角猫猫一个人牺牲。女主角宁愿猫猫补救自己,也不想看到猫猫受伤。女主角说的话正是一周目主角没法选的选项,主角猫猫因而被感动,对女主角的好感度有了一个跃升。同时被打动的还有被封印的灵魂们,他们开始努力挣扎,试着抓住【闪电】,不让【闪电】继续伤害女主角和主角猫猫。而这导致了石像像蜡像一样融化消失。原来,石像的力量来源于被封印的人的执念和怨恨,而现在这些人宁愿自己被封印也不想看到有人为他们牺牲,这从根本上消解了石像的力量。石像因而完全消失在了时空之中。
        }
        {10}{
            男主角再次醒来,发现自己正站在医院之中收集资料,上次预约他因为生病没有来。躺在病床上的人并非因为什么石像而昏迷,他是出了意外事故,有人以讹传讹。这下子【传说A】不攻自破,已经不用再收集资料了。他们最后选择使用【传说B】。
        }
        {11}{
            男主角得到了一个中规中矩的成绩。他被第二周目的女主角的自我牺牲所感动,对女主角暗生情愫,并且因为和女主角生活过一段时间,所以对她格外了解。女主角并不知道男主角变成过猫,但是通过蛛丝马迹她开始有所怀疑。
        }
    }[\PackageError{tree}{Undefined option to tree: #1}{}]%
}

\newcommand{\nvzhuer}[1]{
    \IfEqCase{#1}{
        {0}{
            在妹妹事件的最后,老人在消失之前说,他的意识曾经沉沦了很久,知道十几年前才复苏。并且他觉得主角团总是可以遇到灵异事件,这不太正常,或许冥冥之中有幕后的黑手。这是在战斗之中说的,男主角认为他是在离间大家,因而并没有在意。在战斗之后,解决完足球事件他心中的疑惑逐渐变深,因为那颗足球也是十几年前被遗忘的。  
        }
        {1}{
            在女主事件之中,男主角失去了猫猫。他依照自己作为猫的行为方式,和对于猫的记忆寻找猫猫的过去。他阴差阳错找到了女主角家附近,并且意识到自己化身的猫猫和女主角五岁时养的那只非常像。邻居说他记得很清楚,因为女主曾经差点被一个足球打中,是小猫跳起来救了小时候的女主,因而在身上留下了疤痕。但是,猫不可能活15年还是长得一样。是那只猫的后代,还是另有原因?
        }
        {2}{
            男主角记起在程序员造成的骚动之中,他制造的人工智能曾经对每个人类都加以嘲讽,除了女主角。男主角找到程序员,男主角想重新和修复之后的人工智能对话。人工智能表示,他在网络上除了学籍信息和身份信息,查不到任何关于女主角的痕迹,人工智能对她一无所知自然没法嘲讽。
        }
        {3}{
            男主的疑惑加深,女主角最近在准备一个导师交给她的重要任务。平时一直很乐观的的女主角现在变得有些焦虑,男主角因而没有找她询问。男主角舒了一口气,却又不可抑制地开始自己调查女主角,知道有一天身后有一个人走来。原来是程序员,上次主角莫名其妙的问了关于女主角的事情,程序员就记在心里,最近发现男主角频繁查资料,所以就黑了男主角的电脑,发现了他的举动。男主角被迫将这一切告诉给了程序员,程序员说他会保密,并且帮助男主角。
        }
        {4}{
            程序员的确保守了秘密,但是男主角的异常举止却瞒不住舍友。在舍友的逼问下,男主角就算承认暗恋女主也没有说出他对于女主角的怀疑。舍友放过了男主,转身找到程序员,说已经知道了而关于男主角对于女主角的事情,程序员不知道有诈,就说出了男主角对于女主的怀疑。
        }
        {5}{
            到了第二天,男主角就被骑在身上的妹妹摇醒,原来舍友转手就将这件事告诉给了妹妹,这下子除了女主角之外主角团的都知道了。无奈之下,男主角只好找到大家开了一个背着女主角的秘密会议,大家开始商量对策。在会议之中,大家提出了各种的猜测。程序员认为女主角就是幕后黑手,具有强大的灵力,设计了种种事件想要实现她不可告人的秘密。妹妹认为女主角的确有灵力,但是是出于好心,她设计的事件并未伤害到任何人。舍友开玩笑说女主是凉宫春日的类型,潜意识做这一切,都是为了吸引男主角的注意力,被害羞的男主角否决掉。大家反过来问男主角到底是怎么想的,男主角沉默,他自己也不知道。
        }
        {6}{
            随着时间的推进,男主角发现学校之中的人越来越模板化,就像是游戏之中低智能的NPC一样。而唯一保持正常的女主角和女主角认识或者记得的人。
        }
        {7}{
            主角团再次相聚,大家认为这可能是因为女主角心情不好。大家经过讨论之后,舍友的理论占了主流,需要男主角潜伏到女主角身边,旁敲侧击地搞清楚发生了什么并且将女主角哄好。
        }
        {8}{
            开完会的第二天,男主角发现主角团遗忘了昨天的事情,他们“只记得女主角知道的事情”。男主角突然觉得不寒而栗,他不觉得自己是特别的,也不知道自己可以坚持多久。女主角似乎有心事,她约了男主角去了咖啡厅,男主角问了半天女主只是说导师的任务让她觉得苦恼,男主角说要不然放置一段时间导师的任务,好好生活。问清了导师的任务的截止时间,男主角认为女主可以放三天假期。女主角说日常生活十分单调无聊已经不能让她提起兴趣了,男主角想到了变得循规蹈矩的众人,意识到关键点来了,他忍不住脱口而出,问自己可不可以做女主角的男朋友。
        }
        {9}{
            主角一边和女主角甜蜜,一边调查女主角的方方面面。他心中觉得纠结和痛苦,因为他还是欺骗了女主的感情,更让他痛苦的是他逐渐沉迷其中。
        }
        {10}{
            主角一边和女主角甜蜜,一边调查女主角的方方面面。他心中觉得纠结和痛苦,因为他还是欺骗了女主的感情,更让他痛苦的是他逐渐沉迷其中。
        }
        {11}{
            事情开始好转,学校之中的人都慢慢恢复正常,只是他们依旧没有记起女主角的事情。女主角也恢复了优等生的姿态,男主角向她询问导师的任务如何,女主角说她已经觉得没事了。事情解决,男主角却觉得哪里不对。直到他无意之中提到消失的猫猫,女主角想要做出难过的表情却显得十分僵硬,男主角认定女主角在勉强自己。女主角在进行了完美的演讲之后,并不记得自己讲了什么,她好像快进了很多超出她能力的事情。但是在男主角表现出担心的时候,女主角表示只要大家都喜欢这样的自己,又有什么不可以呢?她软弱的时候只要男主角能看到就好了。
        }
        {12}{
            男主角在一次完美的约会之后,觉得十分舒适,但是他随意地和女主角闲聊,说了一句话之后女主角突然面露异样的表情。男主角突然意识到女主角刚才和自己约会也在快进,而自己甚至没有察觉到这一点,作为男主角不合格。男主角想要道歉,却发现女主角再次变得完美,他甚至找不到和软弱的女主角道歉的机会了。
        }
        {13}{
            在恢复正常的世界里,男主角打算独自探寻真相,主角团也不支持。最后在爷爷的帮助下,男主角发现导师有问题,于是【克服了种种困难】来到了导师那里,进行【某种boss战】。
        }
        {14}{
            导师告诉男主角,这是他作为人类利用灵异之力的一个实验,人类如果可以操纵灵异之力,就像是学会使用火焰的原始人一样得到了巨大的飞跃。而女主角是他的第一个试验品,他想要将一个普通人改造成为人类之中的精英。因而他让普通的小女孩转校到北大附小(如果灵异事件只发生在北大,设定小学和中学也在北大之内),并且在他设计的种种事件之下,让女主角以普通人的资质不断地努力成长,最后考上了北大。但是导师并不满意,他想要更进一步,所以他提取了女主角的记忆设计了一些事件(如果没有男主角,那么将会是女主角和猫猫解决民俗学事件),希望女主角继续成长,但是女主角在事件之中表现的能力开始失去成长性,像是触及了资质的上限。导师决定给女主角更大的压力,开始操作整个学校,push女主角进步,可适得其反,反倒让女主角失去了上进心。但是无心插柳柳成荫,没想到男主角的告白竟然让女主角再次爆发出了上进心,在被人爱的时候女主角想要表现得更好,这让女主角突破了自己的上限——这次她无意识地利用灵异力量改造了自己,让自己成为了自己理想之中的超人。导师因而很感激男主角,解释完一切之后,导师认为自己的试验非常成功,但是男主角认为这是不对的。【阐述普通人理论】,男主角最后打破了导师的意识,将不完美的女主角找了回来,两个人的关系更进一步。
        }
    }[\PackageError{tree}{Undefined option to tree: #1}{}]%
}
\newcommand{\nanzhu}[1]{
    \IfEqCase{#1}{
        {0}{
            男主角和女主角选课:【民俗学】。
        }
        {1}{
            舍友早上提议带领男主参观校园。男主接受之后二人在校园浏览。路过【某些地点】(包括石舫)时,舍友讲述灵异故事。一圈游览下来,舍友把【资料】强塞给男主。晚上发生爷爷对话1.
        }
        {2}{
            男主前去上课,因为路线不熟悉差点迟到。转角正好撞到女主。男主忙于去上课,没有注意自己掉了东西。女主捡到了【资料】。男主冲去上课。民俗学课程留了一个小作业,男主见识到了坐在旁边的程序员用超高技术力瞬间写完。 下课之后出门,恰巧遇到女主。男主这才反应过来自己掉了东西。女主表示看过这份资料,觉得很有意思。男主一开始不以为意,但是在热情的解说下渐渐提起了兴趣。女主问道这资料这么详细,相比你做了很多研究吧?男主拖出资料来源于舍友,女主顺势要求见一下舍友。舍友这时突然出现(“我全都听到了!”),提议成立灵异部。程序员在一旁注意到了此事,但是保持了沉默。
        }
        {3}{
            \hyperlink{meisheyou}{妹妹和舍友的小事件}。舍友顺势邀请妹妹加入社团。晚上,爷爷对话2
        }
        {4}{
            众人在活动室(新太阳?)集合。室友提出要搜集目击情报。男主想到课上令人印象深刻的程序员。下一节课的时候男主鼓起勇气向程序员提出此事,程序员令人意外地一口答应,理由是想要否定灵异事件。
        }
        {5}{
           事件发生。程序员在论坛、树洞上发现石舫传言。女主在系图书馆查到了相关资料。妹妹说她亲眼看到。舍友说他听朋友的朋友的朋友说的(不禁让人怀疑真实性)。总而言之,众人决定前去调查。可能发生之前写过的(问题比较大需要修改的)\hyperlink{huodongshi}{小事件}。调查途中听到路人关于北大人的碎碎念。晚上调查,男主落水。爷爷对话3.
        }
        {6}{
            舍友提议的乱逛。可以设计一些反应主角团其他几人性格的事件(未确定)。遇到和爷爷相似的时间,提升了自己的归属感。遇到路人碎碎念的反转。
        }
        {7}{
            重回石舫,向众人展示自信。最后给爷爷打电话,向他讲述整个故事。爷爷听后笑道看来你在北大也有了自己的生活,以后少给我打电话,多陪朋友们吧。主角也恍然意识到,主角团众人早已是朋友。主角向爷爷的点拨道谢,并表示就算是偶尔也一定会再打电话来的。(可以在后续事件遇到瓶颈的时候再搬出爷爷来,待定。)
        }
    }[\PackageError{tree}{Undefined option to tree: #1}{}]%
}
\newcommand{\ruxue}[1]{
    \IfEqCase{#1}{
        {0}{
            \hyperlink{ruxue0}{【西门入校,遇燕元燕火,带到宿舍楼下,路上陈述一点】}
            \hypertarget{ruxue0}{}
            “北京大学有很多校门。我选择从西门入校。”
            “灰白的世界,透明的面孔。”
            男主是交换生,入学时不会看到一群人入学的盛况。这一点在后续“文化节”弥补。
        }
        {1}{
        【独自宿舍午睡,回想起一些情景——北大梦和北大行】
北大梦:祖父故事,中学努力,高考失利
北大行:夏令营,独自出走,夜游北大,月下少女
        }        
        {2}{
        【室友登场,带男主去校园北边逛,初见灵异】
小时候,男主曾经来过北大,对于湖光塔影以及石舫上的少女,有着模糊的印象。
未名湖边,眼前景色与童年回忆在脑中交织。
男主将目光投向石舫,突然听到了室友的惊呼——石舫上仿佛凭空多了一座建筑。待室友回过神来,那座建筑已经消失了。
        }        
        {3}{
        【首次上课,旁听民俗学,老师是X教授,课程作业要求搜集传说。女主和程序员露个脸。】
        【室友把资料交给男主】
        男主比较喜欢传说怪谈,所以旁听民俗学,没决定选。
        为什么男主获得了资料?几种解释:
        1、X教授钦定。在男主入学前,X教授注意此人是老同学的孙子。此前灵异已经发生,而男主的身份有利于其解决灵异,因此X教授安排会长成为男主室友,以此一步一步吸引男主投入事件。
        2、室友夜游时首次发现灵异,同时察觉到校内的流言。在校庆筹办会议上反馈此事,X教授联想到四十年前的事件,给予「资料」要求室友调查此事。正好男主民俗学课程与之有关,室友邀请男主帮忙。
        }
        {4}{
        【男主根据室友的资料,在校园中进行调查】
        四十年前事件的资料,以此为驱动力,在学校巡礼,逐渐解锁地图。起初,男主不知道这些资料是“四十年前的”,“校庆的”,仅当作“校园怪谈”。
	第一夜发现石舫的灵异后,学校内也出现了有关其他灵异的传闻,这吸引男主进一步调查。男主将这一活动合理化为“了解北大,完成民俗学作业”。(深层原因:想要见证灵异,从而证明自己融入北大,在几次扑空后这一冲动增强)
        }
        {5}{
        【某日黄昏,静园草坪的花树旁,邂逅女主。两人发现都在民俗学课堂上,男主决定选课。】
        女主和男主进行了类似的调查,“她仿佛在跟踪我”,总是出现在相同的地方。印象是:Clannad琴美“前天是小兔子,昨天是小鹿,今天是你”+樱之诗“樱花树下与凛重逢”
        也可以是男主救了女主:某个灵异导致女主遇到困境,要仔细设计,落水什么的太平凡了;不一定是拯救性命,参考冰菓,把女主从上锁的房间捞出来也是一种救。
        }
      
    }[\PackageError{tree}{Undefined option to tree: #1}{}]%
}


\newcommand{\meimei}[1]{
    \IfEqCase{#1}{
        {0}{
        \begin{enumerate}
            \item 妹妹在社团会议上提到自己在寻找一本书,希望大家帮忙。但她对书名讳莫如深,也不透露自己找书的意图。此外还说,这本书在全世界只有一册,藏在北大图书馆。
            \item 男主记得图书馆将会在校庆日附近开放,希望妹妹再等等。女主则欲言又止。
	        \item 几人离开活动室去食堂吃饭,途经图书馆,发现工地门口的告示:由于施工进度,图书馆开放再次延期。
	        \item 女主讲述了最近在中文系盛传的流言:图书馆早就装修好了,是因为闹鬼才延期开放的。
        \end{enumerate}
        }
        {1}{
        \begin{enumerate}
            	\item 男主回到宿舍,发现室友迟迟未归。躺在床上时突然收到妹妹的短信:“会长约我出来了,说可以带我进图书馆找书,哥哥你太菜了。”
	\item 惊恐而愤怒的男主从床上一跃而起,冲向图书馆。男主隐约看到妹妹站在室友旁边,正打算冲上去理论,突然发现女主也在,才稍微松了口气。妹妹对男主做了个鬼脸。
	\item 几人采用最原始的方式——翻墙,潜入了图书馆。庭院中月色如水,早已翻修一新。室友根据地上工人的鞋印,找到一处未上锁的小门。
	\item 进入图书馆,女主惊叹于新馆的结构,妹妹则直奔二楼。男主问妹妹是否可以透露书名,大家一起找。妹妹支支吾吾地回答这是一本文学类的书。男主感到意外,因为本以为妹妹要找的是写论文用的法律专著。
	\item 女主偷偷打听到了书的名字,并查到这本书存放在地下室,几人开始琢磨如何前往地下室。
	\item 妹妹突然说自己听到了声音,并独自奔向一间阅览室(歌剧魅影)。其他人赶快追了上去,但此时妹妹已不见踪影。
	\item 男主、女主和会长在图书馆中到处寻找妹妹,大声呼喊,但无人回应。寻找过程中,发现了暗门、被堵死的楼道、上锁的门等机关,但仍没有找到妹妹。
	\item 大约半小时后,男主透过百讲的玻璃,看到巨大的黑影。(歌剧魅影)
	\item 男主来到百讲前,发现妹妹神情恍惚但笑容满面,手中拿着一本破旧的书。妹妹说,“北大的地下空间都是联通的”,刚才她从图书馆的地下走到了百讲,还找到了自己想要的书。
        \end{enumerate}
        }
        {2}{
        \begin{enumerate}
            	\item 前一晚图书馆探险过后,妹妹的状态不太对劲(表现得很霸气,又很傻气),男主决定偷偷跟着妹妹。【大地图模式】
	\item 一日之间,妹妹走进了各院系的图书馆。(中文、历史、社会学……)男主与管理员、同学对话,了解妹妹的动向。其中加入和程序员的互动。
	\item 最终,妹妹停在了一间办公室门前。男主发现这正是X教授的办公室。
	\item 待妹妹离开后,男主进入了办公室,向X教授说明“妹妹有点不对劲”,并打听那本书的内容。此时女主恰好来访。
	\item X教授说,妹妹所找的书名为《湖》,是一本“元小说”。以北大为背景,讲述了一个少女寻找一本名为《湖》的诗集的故事。【如果X教授支线达成,会得知此事与40年前的校庆有关,据传是X教授的同学编写的】。 
        \end{enumerate}
        }
        {3}{
            \begin{enumerate}
        	\item 晚上,男主对妹妹仍然不放心。和女主告别后,决定潜伏在图书馆旁边。\ZZ{为什么不放心}
	\item 深夜,妹妹果然出现,并准备翻墙。比男主领先一步冲出来的是同样在旁边等候多时的女主。
	\item 两人追问妹妹的来意,多时妹妹才开口:“昨天一位老者带妹妹进入了图书馆的地下,帮她找到了书,并委托妹妹带一些院系图书馆的书过来。”这位老者自称“图书馆的守墓人”。
	\item 此时,他们听到了老者的声音。在声音的引导下,男主一行人推开了暗门,来到了地下。
	\item 图书馆的地下并非阴暗潮湿,而是一个空寂、明亮的场所。(参考回转企鹅罐09集,中央图书馆天空之孔分室,突出一种超现实感)
	\item 一位老者(幽灵)出现在他们背后,无视了男女主,仅和妹妹对话,让她拿出今天在院系图书馆找到的书。
	\item 幽灵翻看着,情绪愈发激动,批判当今的文化界。“礼崩乐坏”“象牙塔的坍塌”
	\item 幽灵的目光转向男女主,缓缓地说:“图书馆是思想的坟墓。你们身上没有纸张的味道……只有她可以继承我的职业,成为大图书馆的守墓人。”(指妹妹)
	\item 【此处补充幽灵教授的设定】20年代的中文教授,一个有些艺术气质的老教授,有自己的一套美学观念,而且有些固执、守旧(辜鸿铭那种感觉?)。诗集《湖》的作者,但他自己忘记了这件事(书的原型是废名的《桥》)。《湖》这本书因为神力被具象化,变成了幽灵,所以这本书就怎么都找不到了。
	\item 【此处补充地下图书馆的设定】地下图书馆同样是神力的产物,并不是真实存在的。具象化之后,幽灵自认为是“图书馆的守墓人”,不停翻看身边的书,目的是寻找记忆中一本十分重要的书《湖》。对当今时代的文化表示好奇,要求妹妹拿书给他看。认为妹妹和自己气质相合,适合做“守墓人”的继承者。(因为妹妹比较读死书,“身上有纸张的气味”。女主虽然读书多,但比较灵活)
	\item 【此处补充妹的设定】平常一直在看书;缺点是读死书,一根筋不知变通;偶尔表现得强气而霸道,实际上很没底气。这些东西都是最后要成长的。
	\item 男女主答应帮幽灵教授寻找《湖》,女主又和幽灵就当今文化进行了一番嘴炮。(妹妹的矛盾和成长,待补完)
	 \end{enumerate}
        }
    }[\PackageError{tree}{Undefined option to tree: #1}{}]%
    }
\newcommand{\chengxu}[1]{
    \IfEqCase{#1}{
{0}{·在解决前几个事件的过程中,男主隐约发现未名BBS某网友“Pygmalion”与事件有所关联(例如:发布灵异事件的调查贴)
·某日,Pygmalion在BBS发布了下一次灵异事件的预告贴,并运用了诗化的语言【类似“预告函”,例如《柯南》】,吸引了主角团的关注。男主委托程序员调查此人,随后得到反馈:Pygmalion的登录信息过于隐蔽,无法得知其身份。}
{1}{几天后,在社团活动室,女主破解了预告贴,称当夜将发生灵异事件,地点是博雅塔下。程序员突然反驳【冷静而克制地,从逻辑的角度反驳】,随后离开。活动室内的气氛变得尴尬,妹妹戳穿了大家的怀疑:Pygmalion可能是程序员本人。
·男主试图联系程序员,未果。整个下午,男主在学校中游荡,一边寻找程序员,一边思考当夜灵异事件的对策。
·黄昏时刻,男主突然接到女主来电,称BBS上有新情报。发现Pygmalion发布了几首诗歌,再次刷新后,Pygmalion的账号已被注销,所有帖子都被删除了。
·预告的时间逐渐临近,男主来到博雅塔下集合,程序员仍未出现。等待之时,室友突然面露恐惧,向男主展示一条信息——“北大网络中心正遭受黑客攻击”。
·预告之时来临,北大在一瞬间陷入黑暗,网络也全部断连。全校大停电,拉开了当夜灵异事件的帷幕。
·男主忽然发现:“博雅塔的窗户中投出一道光束,照向了计算中心的某扇窗户”/“远处的计算中心楼,有一扇窗户仍然亮着”。几人奔向计算中心。
·摸黑登上楼梯,在迷宫般的楼道里兜兜转转,男主终于到达了那间房间的门口,门缝中透出幽绿的光芒。
·房间内,庞大的主机轰鸣着,显示屏照亮了程序员的脸庞。程序员——以及与他模样相同的全息影像——在房间中央无声地对视着。}

第二部分:程序员视角
·Pygmalion是程序员创造的人工智能。Pygmalion拥有学习功能,BBS上的帖子均是自主发布的。
·程序员的人物分析:
A)少年大学生,不屑与同龄人交流,又不擅与年龄较大的同学交流。因此,孤独是他的核心特质。
B)面对孤独,他选择压抑。让外表变得坚毅冷静,使用“纯粹理性”的理论武装使自己看上去无懈可击。
C)他甚至会将自己想象为一台计算机,他的确这样做了——用代码将自己还原为Pygmalion。但这样做总会有剩余——情感的部分,他将其称为bug。
D)他压抑一切,对成熟的渴望,加入主角团的真实目的,以及最深层的——他想理解情感,他想拥有朋友,他想作为人类找到生存的意义,而非作为一台机器。
·Pygmalion的失控:程序员压抑的部分即是他的愿望。在神力的作用下,Pygmalion的性能发生了跨越时代的变化,拥有了自主意识,解除了程序员的压抑,而且拥有超强的运算能力,成为了“完美形态的程序员”。
·停电:当日,离开部室的程序员意识到Pygmalion出现了异常,来到计算中心进行调查。随后发现,Pygmalion已经失控,而且酝酿着对校园网的入侵。程序员与Pygmalion展开了激烈的黑客大战,程序员在即将落败之际,先一步侵入了学校的电力中枢,切断了全校的供电,阻止了人工智能的自我上传。
·人工智能的诗:情感,代码还原的剩余。程序员最初理解的情感是“一种冲动”,驱使自己做某种事情,而且这样做会有快感。在纯粹理性的原则下,这的确是一种bug,应当被消除。
·嘴炮与成长:Pygmalion已经成长为完美形态的程序员,根据纯粹理性的原则,程序员作为一个未完成的个体,是没有必要存在的。那么,程序员如何找回存在的意义?
A)与神力的关系。Pygmalion的失控是程序员无法解释的。要么承认神力,要么承认自己仍有未知的领域,至少无法纳入现有的理论体系。
B)好奇心?(未知领域)真理?(终极意义)美?完成等于死亡,未完成所以才要生存?
C)生存本身就是意义的来源。
·待补充:仍要强调情感的重要性,最终的解决方法不能太抽象。
    }[\PackageError{tree}{Undefined option to tree: #1}{}]%
}
\newcommand{\debi}[1]{
    \IfEqCase{#1}{
        {0}{
 深夜食堂
* 苏德比:CBD麻辣烫小哥,万事屋型角色,经常一边煮麻辣烫一边与同学攀谈,足不出户便能了解到校园各处发生的故事。【参考此间对一些食堂职工的报道】
* 分歧点:无。无论如何都是男主得到猫,仅作为引入事件。【可以改】
* 时间:序盘与男主事件之间,每天晚上。
        }
{1}{
【地点CBD,大地图,强制触发】
晚上,在麻辣烫摊上见到妹妹。
男主旁听麻辣烫小哥与其他同学聊天,发现关键词【石舫】。
男主感到震惊,试图询问,被小哥岔开话,“同学你是第一次来吗,先排队吧。“
【移动到菜架进行排队】小哥一边问“加不加辣”,一边和男主攀谈。
男主问到石舫有关的事情,小哥用很夸张的口气说“好像在闹鬼”。
谈话突然被打断。食品监督员(风纪委员)李星弥巡逻至此,向小哥问话。
问话结束后,男主的麻辣烫已经煮好了,小哥让男主“明天再聊”。
}


{2}{路过CBD,麻辣烫小哥未出现。但大地图可以有其他事件。}
{3}{再次遇到小哥,小哥称“遇到大事了,事情大到需要亲自出马。”两人交换情报,男主说了石舫事件,小哥透露“宿舍闹鬼”。RPG安乐椅推理,得出结论是猫叫。在宿舍地缝中找到猫。}

	* 再次来到麻辣烫摊位,对方称“你来晚了,石舫好像恢复正常了。”
	* 男主问前天为什么不在,小哥说“遇到大事了,事情大到需要我亲自出马。”
	* 男主追问细节,再次被要求点一份麻辣烫。
	* 男主称“石舫事件自己有参与,可以分享一些内幕”,小哥来了兴致,主动送男主一份麻辣烫,并告诉他稍事等候,收摊后细说。
	* 【AVG演出】两人围着CBD绕圈。小哥说,自己对校园生活很向往,但每天必须按时按点卖麻辣烫,只能通过和同学聊天,了解到各种故事。“我有夜宵,你有故事吗?”
	* 谈起了小哥第二天“”
	* 【推理小游戏,进入地图,人物行走图自动演出,男主和小哥画外音评论】
		* 在Day2的晚上,一群人走到摊位前。男生A告诉小哥,男生宿舍最近在闹鬼。
		* 其他人笑话他胆子小,“一群大汉住的地方,鬼都不去。”A很认真地说,这几天每到半夜两点,都会听到窗外有小孩的哭声。
		* 画外音男主推理——春天,可能是猫叫
		* 小哥说A也是这样猜的,但他去窗台上看,甚至下楼寻找,但并没有发现猫。
		* 小哥表示很好奇,问了其他人猫叫春的事情,有的人表示听说过(细节:思考过于认真,麻辣烫煮过头了)
		* 突然卖关子,说“有一个同学透露了重要的情报,自己也是因为这个情报才提前收摊的。”
	* 作为交换,男主讲起石舫的故事(简略表现)。讲完之后,小哥继续叙述。
	* 【推理小游戏,大地图】
		* 同学B说,自己住在A的同栋。他这几天感觉墙里有奇怪的声音。
		* 画外音男主推理——停暖气,水声
		* 小哥说B其实提到了,不只是水声,还包括抓挠爬骚的声音。
		* 男主组合线索,推理得到结论——前几天停暖气,宿舍旁工人修水管,打开了地下室。有一只猫不小心留在了里面,抓挠水管,发出叫声,导致了这种现象。
	* 小哥感到佩服。此时已经接近2点,他们突然听到了“哭声”。
	* 来到宿舍楼前,循声找去,声音果然从某个门内传来。
	* 小哥神奇地掏出了钥匙。“我在卖麻辣烫之前,其实是修水管的。”
	* 最终男主成功解救了猫。这只猫的身上很脏,但仍然很有精神。
* 后日谈
	* 无论猫事件触发与否,男主都会结识苏德比,此后也可以在这里打探情报。        

    }[\PackageError{tree}{Undefined option to tree: #1}{}]%
}
\newcommand{\jiaoshou}[1]{
    \IfEqCase{#1}{
    {0}{假面诗会
* X教授:女主的本科生导师,校庆筹办委员会顾问(隐藏身份)。中文系教授,文学社的指导老师,主要研究现代诗歌。
* 分歧点:如果触发本支线,男主会提前认识X教授,从而在妹妹事件中主动拜访,了解到妹妹所寻找的书与40年前事件的相关性,进而影响最终解谜。
* 时间:程序员事件和妹妹事件发生前。男女主没有确定恋爱关系。
    }
{1}{湖畔【大地图】
	* 某夜,男主路过未名湖边,发现湖心岛上有一群戴着面具的人。【BGM湖畔-神秘】
	* 仔细观察,发现其中一人似乎是女主,那个人向自己打招呼。
	* 男主走上湖心岛,那人摘下面具,果然是女主。
	* 女主向男主介绍,这些是文学社的同学,正在举行“假面诗会”。
	* 男主尚未搞懂什么是假面诗会,就被旁边的人打发走了。}
{2}{社团活动室。
	* 男主和女主聊起昨天的事情。男主了解到,女主是文学社的社员。
	* “假面诗会”是文学的例行活动:深夜在校园选取一个地方,每个人带着面具和自己的诗作,不署名,让大家找出各首诗的作者。【如果感觉没必要写文学社,也可以设定成“中文系的活动”】
	* 男主突然头脑发热,问女主自己能否参加(追求阶段,创造共同经历)/女主邀请男主参加。
	* 最终,男主决定参与下一次假面诗会。【可设置分歧点】}
{3}{中文系图书馆。	
* 男主从未有过创作经验,为自己一时冲动感到后悔,又很担心自己写不好在女主面前出糗。因此来到图书馆寻找参考资料。(需要吐槽,为什么校图书馆进不去)
	* 男主发现了一本诗集《博雅塔》,出版时间是40年前,集录了当年北大学生的现代诗作品。
	* 翻阅时,男主发现某首诗中的一句正是女主的微信签名,这首诗的作者名为……。男主对这首诗产生了很大的兴趣。
	* 一位老者来到书架旁,似乎是在寻找此书。男主走上前去,对方看到了《博雅塔》的封面,露出了惊喜的表情。
	* 老者打算借阅《博雅塔》,男主拍了几张照片后将书递给了老者。办理借阅手续时,男主发现这位老者正是那首诗的作者。在网上查询,发现此人是中文系的X教授,研究现代诗歌。\Ham{男主应该认识教授,民俗学老师}
	* 男主鼓起勇气,向其询问有关那首诗的事情。老者称这是自己学生时代写的,并将男主带到了自己的办公室。【可以和妹妹事件中的《湖》产生联系,比如《博雅塔》是受一本奇妙的书启发,又像诗又像散文又像小说】
	* 交谈之中,教授询问男主的名字,称其像自己的一位故人。【男主祖父】就在此时,有人上门拜访,男主前去开门,发现竟是女主。
	* 两人都倍感惊讶。女主告诉男主,X教授是自己的本科生导师。男主随即离开了。\Ham{这段可能要改,男主对女主、教授关系不确定}
	}
{4}{艰难的创作过程。男主从女主处得知了下次假面诗会的主题——《人工智能之诗》【为程序员事件埋下伏笔】。在所有人的诗中会混入一首人工智能的诗,需要挑选出来。同时,其他人同意男主参与诗会。妄想(表白诗……)}
{5}{假面之夜。	
* 诗会在静园草坪举行。一开始,主持者介绍此次新增了两人。新来的人需要提交一首署名诗,让大家了解其风格。
	* 诗会开始不久,几首风格明显的诗就被指认了作者。还剩下四首。(当大家统一意见,某首诗属于某个人时,那个人就需要回答是或否。若指认成功,那个人就需要摘下假面。)
	* 寻找女主的诗。激烈的心理活动,最终选择枝。【选对了加好感之类的】
	* 剩余三首,有一首是另一位新增者的。那个人被指认后,缓缓摘下面具。男主发现那竟然是室友。室友坏笑着说,他听到男女主的交谈,感觉有意思就也来参加了。
	* 剩余两首,区分男主和人工智能的诗。最终女主成功指认,某首诗具有X教授的风格,应该是男主的。
	* 众人就”人工智能的诗究竟能否算作诗“展开了激烈的讨论,但后来忽然发现静园草坪上的天空十分美丽,大家都摘下面具,躺在草地上仰望星空。
	* 【若男主指认女主诗成功,解锁谈情说爱环节】女主悄悄对男主说,她指认男主的诗并不只是因为X教授的风格,其实是因为“人类的温暖”。诗言情,技法不是最重要的,她为认识了一个纯真而浪漫的男主感到欣喜。}

    }[\PackageError{tree}{Undefined option to tree: #1}{}]%
}
\newcommand{\sheyou}[1]{
    \IfEqCase{#1}{
        {0}{
            (会长单人视角,类似梦境)会长在收拾东西的时候翻出了一颗积满灰尘的足球。会长看着足球,用手擦了擦,似乎看见了上面写着的一些名字……不禁一时沉浸在回忆之中。
            “当年,我也有过这样的经历啊……”就在此时,他仿佛听见一个声音:“是啊,那时候如此辉煌,现在却落得这么个无人问津的下场……”会长环顾四周,却不见人影。“那时候你不是很后悔吗?你一定也这样想过吧:早知如此,还不如从一开始就不去踢什么足球……”“不!我……”“如果你不这么想的话,就证明给我看!要我说,足球这种东西,还不如不存在!”
        }
        {1}{
            会长去梦里那个找到的地方翻找,果然找到了球。会长的翻找引来了众人的好奇,众人纷纷询问这是什么。会长答道这是一颗足球,众人却惊讶的表示“足球是什么?”会长嘲笑了一番几个人,提议说不如我们几个来玩,于是教给众人规则,众人愉快的边学边踢球。快乐的运动时间之后,会长闲聊时提到曾经还有北大杯的足球比赛,众人却表示北大杯有篮球排球等各个项目,唯独没听说过足球。会长笑道“这不可能”,掏出手机想要找出足球队的战报推送等……却发现遍寻不见。会长慌忙地问了几个路人,却没人听说过足球。(地图互动)这时他回想起昨天的梦境,恍然悟道一切是足球从中作梗。
        }
        {2}{
            当天晚上,会长再次梦见足球。会长在梦中质问足球是不是它搞的鬼,足球答道想要结束“足球并不存在”的情况,就组织起来一场比赛来证明足球是有意义的。会长抱怨道这种情况怎么可能凑齐11人的阵容,足球便说那七人制也行,谅你也凑不起来。会长在纠结中缓缓意识淡去。
        }
        {3}{
           会长第二天来到活动室正想提出“组织一场比赛”,女主却抢先说道“有一只队伍想要踢比赛,找到我们问我们能不能接。”其他三人惊讶。会长顺势高兴地接下,提出要主角团进行足球特训。
        }
        {4}{
            标准热血体育竞技题材剧情。需要参考体育漫画(?
        }
        {5}{
           正式比赛的那天,会长来到球场上,惊讶的发现对方队伍正是自己曾经足球队的队友。队友们似乎也忘记了曾经踢球的时光,大家一起在赛场上享受到了最开始踢球的乐趣。比赛最终可以有(类似clannad的)传给队友/自己射门的选项,影响比赛结局(可以不影响后续剧情,或者是不做BE(待定))
        }
        {6}{
            比赛结束之后,会长躺在草坪上享受着余韵。突然他感觉到不对,一切进行的过于顺利了,自己刚在梦中听到要组织比赛就正好有一支队伍,队伍成员还碰巧都是以前足球队的……一切显得过于巧合。就在这时主角团出现,带着水/运动饮料(CG,会长躺在地上的视角,周围一圈是主角团,正上方是刺眼的阳光),说道果然最后你还是注意到了,进入主角团视角的回忆。
        }
        {7}{
            会长无奈地笑道真是服了你们了,内心却因为众人而微微有些感动(Cg模糊处理)。不知何时,球场周围人渐渐多了起来,是平常就来踢球的球友们来催促众人“不踢了就赶紧让场子”。会长从躺着一跃而起(镜头旋转起身),“谁说我不踢了?今天我就要踢个痛快!”
            Misc:事件解决后,周围路人的碎碎念也开始提及足球。
        }
        {8}{
        会长问众人道:“那你们是怎么知道组织比赛就能结束神念的呢。”众人表示不知。会长奇怪道那你们为什么要费劲组织比赛呢?众人答道“看你的表情,大家都知道你期盼着再继续踢球啊。”会长意识到无关神念,大家也一直关注着自己。
        (接下来可以在整个猛男落泪,不过可能有点矫情,待定)
        }
    }[\PackageError{tree}{Undefined option to tree: #1}{}]%
}

\newcommand{\sheyouz}[1]{
    \IfEqCase{#1}{
        {0}{
        主角团在第一天踢完球之后就有人感受到了不对劲(暂定女主)。当天召集了除了会长以外的其他人,众人觉得可能是神念从中作梗。在程序员的搜索下,发现足球确实是存在的,只不过在北大的范围内大家都仿佛集体失忆一般。众人不知道解决的办法,一筹莫展。此时女主提出某事,众人一同谋划。
        }
        {1}{
        某事就是策划比赛,女主提出从未见过会长这么“真正的高兴”,他一定是真心喜欢足球,众人应允。天色已晚,众人决定具体细节择日再议。
        }
        {2}{
        第二天女主突然提出有队,众人惊讶(内心:这你没说过啊!)事后女主说道突然想到如果会长所言不假,那么他以前在足球队一定有队友,我们不如把他们找出来,组织一场比赛。众人感觉又被牵着走了,但是大家也仍然积极响应。
        }
        {3}{
        众人为了找到以前的球队成员费尽周折(比如在院系课程下课时在门口踢球观察众人反应,在课程群发足球图片等,可以被赶出教学楼/被提出群聊等)。最终终于找到了以前的球队成员(这部分可以省去一些细节,略写。)
        }

    }[\PackageError{tree}{Undefined option to tree: #1}{}]%
}
    

\setlength\LTleft{-120pt}
\setlength\LTright{0pt}
\begin{longtable}[]{|p{12pt}|p{80pt}|p{68pt}|p{360pt}|}
\hline
周 & 日期            & 事件 & 描述 \\ \hline
\multirow{2}{*}{1} & \date{\DTMdate{\mydate+ 0 }} &  入学(前?)   & \ruxue{0}  \ZZ{开头的时候需要有一个让人眼前一亮的引导性的剧情,比较紧凑的,驱动男主角在文化节之中探索。可以参考一些已有的游戏的开头。}  \\  \cline{3-4}
& \date{\DTMdate{\mydate+ 1 }} &  入学  &  \ruxue{1} \ZZ{下午在做啥} \\  \cline{3-4}
& \date{\DTMdate{\mydate+ 2 }} &  入学  & \ruxue{2}  \\  \cline{3-4}
& \date{\DTMdate{\mydate+ 3 }} &  入学  & \ruxue{3} \akr{123haha}  \\  \cline{3-4}
& \date{\DTMdate{\mydate+ 4 }} &  入学  & \ruxue{4} \\  \hline
\multirow{2}{*}{1} 
& \date{\DTMdate{\mydate+ 5 }} &  入学  & \ruxue{5}  \\  \cline{3-4}
 & \date{\DTMdate{\mydate+ 6 }} &  男主角1  &\Ham{序盘的故事,晚上很充实,白天有点空。这里建议加一个文化节事件,小型的校庆,可以引出校庆、引导男主体验校园生活、揭示室友真实身份、设计配角登场。} \nanzhu{1} \ZZ{等等,逛校园这个理由用太多次了。可以学校给男主角发了一个像是奶茶店集赞卡一样可以进行盖章的入学手册,需要男主以定向越野的方式前往学校的各个地方完成打卡。然后男主角的入学手册上出现了一个找不到的灵异地点之类的。}\akr{定向越野吧。因为仓鼠的事件这重复了}  \Ham{也不一定是定向越野,就是室友提供资料驱动男主逛校园,表面原因可以是课程作业}\\  \hline
\multirow{2}{*}{2} & \date{\DTMdate{\mydate+ 7 }} &  男主角1  & \nanzhu{2} \ZZ{舍友出现的太鬼畜了,换个入场方式;灵异部的成立可以在解决第一个事件之后吗?在大家解决完事件心情畅快的时候,决定将主角团固化下来成为一个社团。} \akr{我觉得可以。前面有实无名即可}\Ham{那其实主角团集合就不急了,妹妹和程序员可以慢慢登场。}\\  \cline{3-4}
 & \date{\DTMdate{\mydate+ 8 }} &  男主角1  & \nanzhu{3}\Ham{妹妹可以叙诡,前面有男主和一个人发消息的桥段,以及约某个人吃饭的桥段,但隐藏对方的身份。}  \\  \cline{3-4}
 & \date{\DTMdate{\mydate+ 9 }} &  男主角1  & \nanzhu{4} \ZZ{建议这里增加一些关于程序员的冲突} \Ham{比如重新沿用全息投影的设定,有一天男主突然能看到石舫上的影像,但舍友连摇头“是假货”,最后发现程序员拿着投影仪搞事情,顺势拉入伙。“我想在湖心岛拉他入伙,而不是教室。”}  \\  \cline{3-4}
 & \date{\DTMdate{\mydate+ 10 }} &  男主角1  & \nanzhu{5} \ZZ{爷爷对话有没有一个比较固定的触发机制,比如看到爷爷日记本之类的}\akr{我设计的是打电话来。}\Ham{我以为爷爷去世了。} \\  \cline{3-4}
 & \date{\DTMdate{\mydate+ 10 }} &  苏德比  & \debi{1} \\  \cline{3-4}
 & \date{\DTMdate{\mydate+ 11 }} &  男主角1  & \nanzhu{6} \\  \hline
 & \date{\DTMdate{\mydate+ 11 }} &  苏德比  & \debi{2} \\  \hline
\multirow{2}{*}{2} & \date{\DTMdate{\mydate+ 12 }} &  男主角1  & \nanzhu{7} \Ham{重要的转折点是:从看不见到看见,然后如何解决?在石舫上欢饮达旦,然后成立灵异部。(想个好听的名字)} \\  \cline{3-4}
 & \date{\DTMdate{\mydate+ 13 }} &  初遇燕元燕火 & 暂定,需要一个相遇事件\Ham{我设定一开始就遇到燕元燕火了,这里可以再次遇到}    \\  \hline
 & \date{\DTMdate{\mydate+ 13 }} &  苏德比 & \debi{3}  \ZZ{这里是把小哥设定为一个掮客一样的角色吗?那可以描绘一个“里北大”加深一下气氛。}\Ham{原型是深夜食堂,里北大可以有,那德比性格就更神秘一点。} \\  \hline
% dream start
\multirow{2}{*}{\textcolor{red}{3}} & \date{\textcolor{red}{\DTMdate{\mydate+ 14 }}} &  女主  & \nvzhumeng{1}  \\  \cline{3-4}
 & \textcolor{red}{\date{\DTMdate{\mydate+ 15 }}} &  女主  & \nvzhumeng{2}  \\  \cline{3-4}
 & \textcolor{red}{\date{\DTMdate{\mydate+ 16 }}} &  女主  & \nvzhumeng{3} \Ham{考证一下,尽量中式恐怖一点} \\  \cline{3-4}
 & \textcolor{red}{\date{\DTMdate{\mydate+ 17 }}} &  女主  & \nvzhumeng{4} \Ham{40年是不是有点夸张} \\  \cline{3-4}
 & \textcolor{red}{\date{\DTMdate{\mydate+ 18 }}} &  女主  & \nvzhumeng{5}  \\  \hline
\multirow{2}{*}{\textcolor{red}{3}} & \textcolor{red}{\date{\DTMdate{\mydate+ 19 }}} &  女主  & \nvzhumeng{6} \Ham{可以把山洞改成校园内某处吗?北大的地下在后续剧情中用得到,可以考虑一下。比如李兆基人文学院,一个始终不开门的院子,实际上镇压着咒物。} \ZZ{这个可以;或者换个校区} \\  \cline{3-4}
 & \textcolor{red}{\date{\DTMdate{\mydate+ 20 }}} &  女主  & \nvzhumeng{7}  \\  \hline
\multirow{4}{*}{\textcolor{red}{4}} & \textcolor{red}{\date{\DTMdate{\mydate+ 21 }}} &  女主  & \nvzhumeng{8}  \\  \cline{3-4}
 & \textcolor{red}{\date{\DTMdate{\mydate+ 22 }}} &  女主  & \nvzhumeng{9}  \\  \cline{3-4}
 & \textcolor{red}{\date{\DTMdate{\mydate+ 23 }}} &  女主  & \nvzhumeng{10}  \\  \cline{3-4}
 & \textcolor{red}{\date{\DTMdate{\mydate+ 24 }}} &  女主  & \nvzhumeng{11}  \\  \cline{3-4}
 \hline
% dream end
\multirow{2}{*}{3} & \date{\DTMdate{\mydate+ 14 }} &  女主  & \nvzhu{1}\Ham{另一种想法:男主被雕像封印,猫猫前来拯救。故事在高潮部分戛然而止,猫可能会死亡,男主灵魂可能会得救,留个悬念。恢复意识时,男主直接变猫了。
时间并没有发生回溯。变猫的男主寻找自己的身体,最终发现自己在校医院昏迷不醒,妹妹、室友在旁照顾(凉宫春日的消失结尾)。猫听讨论得知女主不见了,突然意识到女主陷于危险之中,准备拯救女主。
猫找到了女主,此时女主通过翻找男主的资料,发现男主昏迷和雕像有关,正在校园中寻找。猫纠结要不要告诉女主雕像所在地。(只有女主能救男主,但救男主导致女主置于危险。矛盾)最终决定独自返回地宫,迎战雕像。没想到女主察觉到猫的奇怪,追了过来。反复摆脱未果,两人来到地宫。
战斗最后,猫决定自我牺牲,但被女主的智慧拯救了。智慧:雕像抓人设计一个规则,类似鬼捉人,比较有逻辑性,这样就可以最后找漏洞。比方说一次只能抓一个,女主主动出来当替死鬼,然后blablabla。自身产生自身的闭环递归,前目的地}  \\  \cline{3-4}
 & \date{\DTMdate{\mydate+ 15 }} &  女主  & \nvzhu{2}  \\  \cline{3-4}
 & \date{\DTMdate{\mydate+ 16 }} &  女主  & \nvzhu{3}  \\  \cline{3-4}
 & \date{\DTMdate{\mydate+ 17 }} &  女主  & \nvzhu{4}  \\  \cline{3-4}
 & \date{\DTMdate{\mydate+ 18 }} &  女主  & \nvzhu{5}  \\  \hline
\multirow{2}{*}{3} & \date{\DTMdate{\mydate+ 19 }} &  女主  & \nvzhu{6}  \\  \cline{3-4}
 & \date{\DTMdate{\mydate+ 20 }} &  女主  & \nvzhu{7}  \\  \hline
\multirow{2}{*}{4} & \date{\DTMdate{\mydate+ 21 }} &  女主  & \nvzhu{8} \Ham{我觉得这段剧情有点牵强。女主和男主刚认识不久,男主角仅仅是大作业受挫,女主就这么做有点倒贴的感觉,会失去后续攻略的成就感。要这么写的话,前面还是让男主角遇到一般意义上的危险,使男主的悲惨事件和女主的自我牺牲对等,否则很难让大学生之外的群体产生共情。} \ZZ{按照阿卡林的说法,这个女人就是这么傻,这么舔(x} \\  \cline{3-4}
 & \date{\DTMdate{\mydate+ 22 }} &  女主  & \nvzhu{9}  \\  \cline{3-4}
 & \date{\DTMdate{\mydate+ 23 }} &  女主  & \nvzhu{10}  \\  \cline{3-4}
 & \date{\DTMdate{\mydate+ 24 }} &  女主  & \nvzhu{11} \ZZ{稍微体现一下女主角的中央空调行为。} \\  \cline{3-4}
 & \date{\DTMdate{\mydate+ 25 }} & 程序员  & \chengxu{0}  \\  \hline
\multirow{2}{*}{4} & \date{\DTMdate{\mydate+ 26 }} &  教授  & \jiaoshou{1} \\  \cline{3-4}
 & \date{\DTMdate{\mydate+ 27 }} &  教授  & \jiaoshou{2} \\  \hline
\multirow{2}{*}{5} & \date{\DTMdate{\mydate+ 28 }} &  教授  & \jiaoshou{3}  \\  \cline{3-4}
 & \date{\DTMdate{\mydate+ 29 }} &  教授  & \jiaoshou{4}  \\  \cline{3-4}
 & \date{\DTMdate{\mydate+ 30 }} &  程序员  & \chengxu{1} \ZZ{这个故事我看了看,感觉不够曲折,这种幕后黑手是自己的第二重人格/人工智能的桥段任何读者都一眼能看穿。我建议加入如下叙述性诡计:首先程序员没有多重人格或者是多重人格已经相互和解,真正因为多重人格而导致有毛病的是人工智能Pygmalion; 所有程序员视角的部分转换为名字隐藏的第一人称视角,本质上是Pygmalion的感性视角(人格),Pygmalion的感性人格认为自己真的是一个人,不认同Pygmalion的理性人格所做的事情(这些事情保持不变),并且尝试和它进行斗争,在他斗到最后,看到了主角团闯入。这时候安排第三部分的解密环节,从匿名(Pygmalion的感性人格)视角切换为主角视角,真正的程序员出现,点出Pygmalion的感性视角也不是真人,真人会如何平衡自己的感性和理性,并且自己没有多重人格。就此Pygmalion的两个人格被程序员本人真正解决(如何解决,讨论的问题保持不变),主角团露出了开心的笑容,却没有发现程序员被屏幕照得半黑半白的脸一半面无表情一半在笑(暗示程序员本人可能依旧有其他人格只是并未显现)}\Ham{这个故事已经设计了一层反转,前面诱导玩家怀疑是程序员第二人格,后面揭示是人工智能,从解谜的角度来说是足够的;故事设计为程序员和人工智能的和解,实际上是和自我的和解,前面会讲述程序员的早期经历,人工智能即是他的“理想人格”(Pygmalion),然而这种理想亦是扭曲的。人工智能的自我扭曲也可以套用以上故事,不过无法体现程序员的成长,而且最后“半明半暗”暗示程序员后续还有故事,程序员自身的问题未解决。}\akr{我觉得这里的老套没问题。揭露犯人可在悬疑一点,现在的诡计持续时间有点短,导致故事显得单调} \ZZ{建议这个故事应该向着一种高级感走。} \\  \cline{3-4}
 & \date{\DTMdate{\mydate+ 31 }} &  教授  & \jiaoshou{5}  \\  \cline{3-4}
 & \date{\DTMdate{\mydate+ 32 }} &  妹妹  & \meimei{0} \\  \hline
\multirow{2}{*}{5} & \date{\DTMdate{\mydate+ 33 }} &  妹妹  & \meimei{2}  \\  \cline{3-4}
 & \date{\DTMdate{\mydate+ 34 }} &  妹妹  & \meimei{3}  \\  \hline
\multirow{2}{*}{6} & \date{\DTMdate{\mydate+ 35 }} &风纪委员日常  & \hyperlink{fengjirichang}{风纪委员日常} \ZZ{这个故事缺乏一定的推动力,妹妹跑来跑去是为什么?需要一个推动的理由,元小说对妹妹有什么帮助和好处?妹妹对于自己目前的生活有什么不满足。这个故事追求少女感,新房感。} \\  \cline{3-4}
 & \date{\DTMdate{\mydate+ 36 }} &  舍友  & \sheyou{0}  \\  \cline{3-4}
 & \date{\DTMdate{\mydate+ 37 }} &  舍友  & \sheyou{1}  \\  \cline{3-4}
  & \date{\DTMdate{\mydate+ 37 }} &  舍友  & \sheyouz{0} \Ham{校内所有人的记忆都被改写了,那校外呢?互联网呢?这种记忆改写,希望他人相信的剧情,可以参考石头门。} \\  \cline{3-4}
  & \textcolor{red}{\date{\DTMdate{\mydate+ 37 }}} &  舍友  & \sheyouz{1} \akr{这个红字日期表示在回忆部分才出现的倒叙,下同} \\  \cline{3-4}
 & \date{\DTMdate{\mydate+ 37 }} &  舍友  & \sheyou{2} \Ham{中间得有点别的剧情,燕元燕火支线。另外校庆快到了,会长不能全心全意的踢足球,这方面可以设置为阻力,“一天只能练2小时。”} \\  \cline{3-4}
  & \date{\DTMdate{\mydate+ 38 }} &  舍友  & \sheyou{3} \akr{这里需要加入一个凑人环节,暂定想起了风纪委员和麻辣烫的苏德比。目前还没写}  \\  \cline{3-4}
   & \textcolor{red}{\date{\DTMdate{\mydate+ 38 }}} &  舍友  & \sheyouz{2}  \\  \cline{3-4}
 & \date{\DTMdate{\mydate+ 39 }} &  舍友  & \sheyou{4}  \\  \hline
 & \textcolor{red}{\date{\DTMdate{\mydate+ 39 }}} &  舍友  & \sheyouz{3}  \\  \cline{3-4}
\multirow{2}{*}{6} & \date{\DTMdate{\mydate+ 40 }} &  舍友  & 舍友发现众人格外疲惫,但是没太在意,依旧训练。此段剧情我感觉可能需要数天,但是目前没想好具体时间线。  \\  \cline{3-4}
 & \date{\DTMdate{\mydate+ 41 }} &  舍友  & \sheyou{5}  \\  \hline
\multirow{2}{*}{7} & \date{\DTMdate{\mydate+ 42 }} &  舍友  & \sheyou{6} \ZZ{这里没看懂}  \\  \cline{3-4}
& \date{\DTMdate{\mydate+ 42 }} &  舍友  & \sheyou{7} \\  \cline{3-4}
& \date{\DTMdate{\mydate+ 43 }} &  舍友  & \sheyou{8}  \\  \cline{3-4}
 & \date{\DTMdate{\mydate+ 44 }} &  舍友  & 这里要多两天凑人  \\  \cline{3-4}
 & \date{\DTMdate{\mydate+ 45 }} &  舍友  & 这里要多两天凑人  \\  \cline{3-4}
& \date{\DTMdate{\mydate+ 46 }} &  舍友  & 这里要多一天热血体育,这样差不多也是两周  \\  \hline
\multirow{2}{*}{7} & \date{\DTMdate{\mydate+ 47 }} &  小事件  &  \\  \cline{3-4}
 & \date{\DTMdate{\mydate+ 48 }} &  小事件  &   \\  \hline
\multirow{2}{*}{8} & \date{\DTMdate{\mydate+ 49 }} &  小事件  &   \\  \cline{3-4}
& \date{\DTMdate{\mydate+ 50 }} &  女主2  & \nvzhuer{0}  \\  \cline{3-4}
 & \date{\DTMdate{\mydate+ 51 }} &  女主2  & \nvzhuer{1}  \\  \cline{3-4}
 & \date{\DTMdate{\mydate+ 52 }} &  女主2  & \nvzhuer{2}  \\  \cline{3-4}
 & \date{\DTMdate{\mydate+ 53 }} &  女主2  & \nvzhuer{3}  \\  \hline
\multirow{2}{*}{8} & \date{\DTMdate{\mydate+ 54 }} &  女主2  & \nvzhuer{4}  \\  \cline{3-4}
 & \date{\DTMdate{\mydate+ 55 }} &  女主2  & \nvzhuer{5}  \\  \hline
\multirow{2}{*}{9} & \date{\DTMdate{\mydate+ 56 }} &  女主2  & \nvzhuer{6}  \\  \cline{3-4}
 & \date{\DTMdate{\mydate+ 57 }} &  女主2  & \nvzhuer{7}  \\  \cline{3-4}
 & \date{\DTMdate{\mydate+ 58 }} &  女主2  & \nvzhuer{8}  \\  \cline{3-4}
 & \date{\DTMdate{\mydate+ 59 }} &  女主2  & \nvzhuer{9}  \\  \cline{3-4}
 & \date{\DTMdate{\mydate+ 60 }} &  女主2  & \nvzhuer{10}  \\  \hline
\multirow{2}{*}{9} & \date{\DTMdate{\mydate+ 61 }} &  女主2  & \nvzhuer{11}  \\  \cline{3-4}
 & \date{\DTMdate{\mydate+ 62 }} &  女主2  & \nvzhuer{12}  \\  \hline
\multirow{2}{*}{10} & \date{\DTMdate{\mydate+ 63 }} &  女主2  & \nvzhuer{13}  \\  \cline{3-4}
 & \date{\DTMdate{\mydate+ 64 }} &  女主2  & \nvzhuer{14}  \\  \cline{3-4}
 & \date{\DTMdate{\mydate+ 65 }} &  男主角入学  & \hyperlink{event:nzj}{这是男主角入学的事件}  \\  \cline{3-4}
 & \date{\DTMdate{\mydate+ 66 }} &  男主角入学  & \hyperlink{event:nzj}{这是男主角入学的事件}  \\  \cline{3-4}
 & \date{\DTMdate{\mydate+ 67 }} &  男主角入学  & \hyperlink{event:nzj}{这是男主角入学的事件}  \\  \hline
\multirow{2}{*}{10} & \date{\DTMdate{\mydate+ 68 }} &  男主角入学  & \hyperlink{event:nzj}{这是男主角入学的事件}  \\  \cline{3-4}
 & \date{\DTMdate{\mydate+ 69 }} &  男主角入学  & \hyperlink{event:nzj}{这是男主角入学的事件}  \\  \hline
\multirow{2}{*}{11} & \date{\DTMdate{\mydate+ 70 }} &  男主角入学  & \hyperlink{event:nzj}{这是男主角入学的事件}  \\  \cline{3-4}
 & \date{\DTMdate{\mydate+ 71 }} &  男主角入学  & \hyperlink{event:nzj}{这是男主角入学的事件}  \\  \cline{3-4}
 & \date{\DTMdate{\mydate+ 72 }} &  男主角入学  & \hyperlink{event:nzj}{这是男主角入学的事件}  \\  \cline{3-4}
 & \date{\DTMdate{\mydate+ 73 }} &  男主角入学  & \hyperlink{event:nzj}{这是男主角入学的事件}  \\  \cline{3-4}
 & \date{\DTMdate{\mydate+ 74 }} &  男主角入学  & \hyperlink{event:nzj}{这是男主角入学的事件}  \\  \hline
\multirow{2}{*}{11} & \date{\DTMdate{\mydate+ 75 }} &  男主角入学  & 校庆当天,男主受到舍友夺命电话,说校庆上发生了恶劣事件,需要帮忙。男主连忙前往校庆。

结果遇到了预告函:前往各地,连环解谜,指向下一个位置。

遍历已经触发的各个大小事件地点。系统配合跑图。可以有一些解谜小游戏。
最终谜题:“谁是这次解谜的幕后黑手?前往初次相遇之地”。分支:
\begin{itemize}
    \item 是舍友-宿舍。舍友线。
    \item 是妹妹-不知道该去哪儿,四处转了转无果。跳到活动室过生日剧情。
    \item 是程序员-去了但没找到人。自知答案错了。
    \item 是女主-前去【初遇女主的地方】。恋爱线。
    \item 都不是-自由行动。存在若干分支:
    \begin{itemize}
        \item 若去活动室:发现众人正在准备给男主过生日(设定)。可以选择离开或者推门进入。离开则可以去其他地方。进入则看到未准备好的众人,且舍友、女主不在。
        \item (隐藏路线?)在新太阳一个角落找到风纪委员。
        \item 去麻辣烫:买了麻辣烫。自己吐槽“肯定不是他吧”

    \end{itemize}
         最终都会回到活动室,遇到众人给自己过生日。
\end{itemize}
      \\  \cline{3-4}
 & \date{\DTMdate{\mydate+ 76 }} &  男主角入学  & \hyperlink{event:nzj}{这是男主角入学的事件}  \\  \hline
\multirow{2}{*}{12} & \date{\DTMdate{\mydate+ 77 }} &  男主角入学  & \hyperlink{event:nzj}{这是男主角入学的事件}  \\  \cline{3-4}
 & \date{\DTMdate{\mydate+ 78 }} &  男主角入学  & \hyperlink{event:nzj}{这是男主角入学的事件}  \\  \cline{3-4}
 & \date{\DTMdate{\mydate+ 79 }} &  男主角入学  & \hyperlink{event:nzj}{这是男主角入学的事件}  \\  \cline{3-4}
 & \date{\DTMdate{\mydate+ 80 }} &  男主角入学  & \hyperlink{event:nzj}{这是男主角入学的事件}  \\  \cline{3-4}
 & \date{\DTMdate{\mydate+ 81 }} &  男主角入学  & \hyperlink{event:nzj}{这是男主角入学的事件}  \\  \hline
\multirow{2}{*}{12} & \date{\DTMdate{\mydate+ 82 }} &  男主角入学  & \hyperlink{event:nzj}{这是男主角入学的事件}  \\  \cline{3-4}
 & \date{\DTMdate{\mydate+ 83 }} &  男主角入学  & \hyperlink{event:nzj}{这是男主角入学的事件}  \\  \hline
\multirow{2}{*}{13} & \date{\DTMdate{\mydate+ 84 }} &  男主角入学  & \hyperlink{event:nzj}{这是男主角入学的事件}  \\  \cline{3-4}
 & \date{\DTMdate{\mydate+ 85 }} &  男主角入学  & \hyperlink{event:nzj}{这是男主角入学的事件}  \\  \cline{3-4}
 & \date{\DTMdate{\mydate+ 86 }} &  男主角入学  & \hyperlink{event:nzj}{这是男主角入学的事件}  \\  \cline{3-4}
 & \date{\DTMdate{\mydate+ 87 }} &  男主角入学  & \hyperlink{event:nzj}{这是男主角入学的事件}  \\  \cline{3-4}
 & \date{\DTMdate{\mydate+ 88 }} &  男主角入学  & \hyperlink{event:nzj}{这是男主角入学的事件}  \\  \hline
\multirow{2}{*}{13} & \date{\DTMdate{\mydate+ 89 }} &  男主角入学  & \hyperlink{event:nzj}{这是男主角入学的事件}  \\  \cline{3-4}
 & \date{\DTMdate{\mydate+ 90 }} &  男主角入学  & \hyperlink{event:nzj}{这是男主角入学的事件}  \\  \hline
\multirow{2}{*}{14} & \date{\DTMdate{\mydate+ 91 }} &  男主角入学  & \hyperlink{event:nzj}{这是男主角入学的事件}  \\  \cline{3-4}
 & \date{\DTMdate{\mydate+ 92 }} &  男主角入学  & \hyperlink{event:nzj}{这是男主角入学的事件}  \\  \cline{3-4}
 & \date{\DTMdate{\mydate+ 93 }} &  男主角入学  & \hyperlink{event:nzj}{这是男主角入学的事件}  \\  \cline{3-4}
 & \date{\DTMdate{\mydate+ 94 }} &  男主角入学  & \hyperlink{event:nzj}{这是男主角入学的事件}  \\  \cline{3-4}
 & \date{\DTMdate{\mydate+ 95 }} &  男主角入学  & \hyperlink{event:nzj}{这是男主角入学的事件}  \\  \hline
\multirow{2}{*}{14} & \date{\DTMdate{\mydate+ 96 }} &  男主角入学  & \hyperlink{event:nzj}{这是男主角入学的事件}  \\  \cline{3-4}
 & \date{\DTMdate{\mydate+ 97 }} &  男主角入学  & \hyperlink{event:nzj}{这是男主角入学的事件}  \\  \hline
\multirow{2}{*}{15} & \date{\DTMdate{\mydate+ 98 }} &  男主角入学  & \hyperlink{event:nzj}{这是男主角入学的事件}  \\  \cline{3-4}
 & \date{\DTMdate{\mydate+ 99 }} &  男主角入学  & \hyperlink{event:nzj}{这是男主角入学的事件}  \\  \cline{3-4}
 \hline
\end{longtable}


% \section{素材}

% 一个专门用来讲男女主恋爱的大事件。比如一个猫的事件。

Q:如何让玩家对男女主的爱情有代入感?如何承认官方CP?
A:换视角,一种更加高级的换视角。让玩家可以同时从男女主的角度来看待对方。最好让玩家在女主角的视角下觉得男主角很温柔可爱,必须要把男主角娶到手。

Q:一般来说女主角是不可操纵角色。如何以此为前提,在只操作男主的情况下换视角?
A:(我从起点流行的游戏化写作学到的一招)只换角色的身份,不换角色本身。首先主角和一只猫遭遇了神秘事件,并且在事件之中遇到危险,猫猫对主角不离不弃,傻傻地帮助主角同时卖萌,主角对猫猫很喜爱,要是人的话甚至想娶了她。但是主角在这个“游戏”之中只打到了normal end,时间解决的不完整,猫猫为了主角牺牲了,主角抱着猫猫痛哭。但是因为事件没解决完美,主角变成了猫猫,同时遇到了女主,并且陷入了相通的时间。有了一周目记忆的主角想要避免normal end的失败,但是是猫猫没法说话,只能一边卖萌一边想办法引导女主角,。最后在女主对于猫猫的牺牲下,(插入:女主说猫猫你救了我这么多次,这次换我救你了),达成happy end。好处是玩家可以把一周目对于猫猫的爱共情到女主对于二周目主角的爱,这样就很合理。女主角对猫猫吐露心声也更加真实。两个周目实际上是对一个故事的两次叙述,可以埋更多的伏笔,也是对一个恋爱的双方视角的阐述——主角在一周目实际上是“一般故事主角视角”,是被猫猫爱的一方,在这个周目感受猫猫(男主角)的可爱,解释为什么女主会爱男主;主角在二周目实际上是“一般故事配角视角”,扮演一只猫猫去爱一个“一般故事主角”,在这个周目感受一般故事主角(女主角)的可爱(因为女主角反过来救了配角),解释主角为什么会喜欢女主角。
% % \section{恋爱大事件}
有折叠内容。
第一周目:【男主角】做了一个梦。在【日期A】, 他遇到了一只可爱的猫猫。猫猫有点不听话,总是做出一些匪夷所思的举动,但是主角还是很喜欢这只猫猫。主角在一时兴起,准备参加在【日期B】举办北大举办的【某个比赛】。主角自己拿到冠军的希望并不大,但是他还有猫猫帮他。最后在猫猫提供的【一些帮助】下,主角阴差阳错地拿到了冠军,但是猫猫却受到了【某种伤害】。在拿到冠军之后,主角已经开始有些后悔了,这时候遇到了同样参加比赛的【女主角】。女主角透露了自己想要参加比赛的【某种原因】,并且她所付出的努力,后最后收获的失望,并且露出了强颜欢笑的表情祝福主角。主角意识到自己根本不应该拿这个冠军,甚至为了这个冠军害了猫猫,也让女主角很收获了遗憾。他下定决心如果时间可以重来,那么他就一定要放弃这个冠军。</summary><blockquote>
【某个比赛】:一个女主角很重要的比赛,比赛里有运气的成分。*改为【校庆活动策划招标】,方案是否中标有一定评委因素和运气成分。*
【一些帮助】:三种帮助,需要用猫猫的身体跑地图完成。
【受到某种伤害】:猫猫受到的伤害一开始觉得不大,第二周目才觉得大*改为联想到女主会做同样的牺牲。体现在猫身上,可以是废寝忘食帮男主做事,最后饿晕或生病等。*
【某种原因】:适当沉重,要有分量*改为付出的努力没有得到认可,心情失落*
<details><summary>比赛报名:主角醒来,发现自己在宿舍,窗台处进来了一只猫。主角注意到今天是【日期A】,他一时兴起打算参加某个比赛。主角在比赛前因为【某些原因】和别人打赌,一定要获得胜利。</summary></details>

<details><summary>比赛前期:主角被比赛难倒了,他随便瞎蒙了一下*猫为他提供了灵感*,运气爆棚,竟然意外地晋级了复赛。猫猫似乎也因此而高兴,主角开始扬眉吐气。</summary>
</details>

<details><summary>比赛中期:主角在比赛之中顺风顺水,开始洋洋得意。同时猫猫和主角的日常生活之中发生了一些趣事,让主角对猫猫越发地喜欢。</summary></details>

<details><summary>比赛后期:主角的运气终于用尽,他担心自己吹出去的牛没法兑现,开始执迷于冠军。这时候他想到了他通人性的猫猫,意识到可以让猫猫合法作弊。他指挥猫猫,最后猫猫【受到某种伤害】的同时满足了他的愿望,让他拿到了冠军。</summary>
</details>

<details><summary>主角后悔期:主角在高兴了一段时间之后,觉得有些空虚。他没有为冠军付出过努力,也不怎么把它当一回事,得到了太轻松,事后也觉得没啥意思。这时才意识到猫猫收到了的伤害,他开始变得自责。后来他遇到了女主角,了解了她的故事后更发现,这个冠军是自己从她那里偷来的。*发生了一些小事,让男主意识到猫猫的牺牲正是女主也非常可能做出的。*</summary></details>

<details><summary>主角“穿越”:主角在返回后,希望回到自己报名的那一天,让自己根本不去报名,放弃参加这个比赛。他的愿望被【某个存在(可以是猫猫,或者是别的什么)】发现并且实现了。*男主此时并没意识到为什么自己穿越了*</summary>
<blockquote>【某个存在】:设想的是猫猫。一周目的时候猫猫一直在尽量实现主角的愿望,这时候听到男主角想要重来的期望后,决定将男主角的意识承载在自己身上然后进行穿越。在二周目猫猫的意识实际上一直和男主角的意识同时存在。*可能是男主自己的愧疚感。自己的成就是基于他人的牺牲,这让男主想要重来*</blockquote>
</details>
</blockquote></details>

</blockquote></details>
<details><summary>第二周目:主角醒来,发现之前的一切似乎是一个梦,而自己又重新回到了【日期A】,只是他不再是人而是那只猫猫。变成猫猫的主角找到了自己的原身,想要阻止他报名比赛,但是已经晚了。主角意识到,对于原身来说“放弃拿到冠军”很简单,但是对于不能说话的猫猫来说却千难万难。*改为让男主放弃向女主求助,想让男主独力完成,同时阻止女主的自我牺牲。*主角打算学习一周目的猫猫,帮助女主角击败自己拿到冠军。他试着给女主角提供猫猫一周目给过他的【一些帮助】,却发现原来一周目的猫猫为了他克服了那么多的困难。就算是有着一周目的记忆,知道部分未来,但依旧千难万难。这时候他的愧疚心愈发膨胀。同时执意参加比赛的原身成了主角的大敌,主角发现原来过去的自己是如此的幼稚而可恶。到了比赛当天的【日期B】,主角发现为了让女主角拿到冠军,唯一的方法就是牺牲自己。像一周目的猫猫那样承受【某种伤害】。主角认为冠军本来就应该是女主角的,因此打算效仿一周目的猫猫自我牺牲。但是这时候女主角发现了化身为猫猫的男主的打算,她愿意放弃比赛,也不想让猫猫受伤。主角深受感动,对女主角的好感度有了一个跃升。最后因为他们相互牺牲的行为达成了【某些条件】,导致事情走向了happy end,猫猫没有受伤,女主角也拿到了冠军。</summary><blockquote>
【某些条件】:???
<details><summary>比赛报名:主角醒来,发现自己竟然站的回到了【日期A】。但是高兴之余,他发现自己不再是人,而是那只猫猫。他像一周目的猫猫一样找到了主角,想要阻止原身一时兴起参加比赛,但是失败了。猫猫不能说话,本以为很简单的事情却意外地复杂。</summary></details>

<details><summary>比赛前期:猫猫找到了女主角,打算帮助女主角获得胜利。它和女主角发生了一周目发生过的【一些趣事】。主角这才发现,在猫猫的视角来看,这些事件包含了猫猫对主角的爱。主角对一周目猫猫好感度大幅上升。</summary>
<blockquote>【一些趣事】:要可以增加男女主角的感情,同时揭露女主角性格和习惯。
</blockquote></details>

<details><summary>比赛中期:原身在比赛之中顺风顺水,开始洋洋得意。女主角在比赛之中开始受挫,在猫猫的暗中帮助之下才勉强过关。主角意识到一周目里让自己洋洋得意的运气根本不是什么好东西。在他现在的视角来看竟然是如此的可恶。果然所有命运赠送的礼物,早已在暗中标好了价格。</summary></details>

<details><summary>比赛后期:原身的运气终于用尽,但是他不愿意放弃。女主角在比赛后期被不愿意放弃的原身干扰,拿到冠军的可能性也变得渺茫。主角认为如果原身不参赛,女主角可以很顺利地拿到冠军。他意识到唯一可以让事情回到正轨的方法是像一周目的猫猫那样帮助女主角。</summary></details>

<details><summary>相互拯救:主角在学习猫猫的时候,才发现一周目的猫猫到底为自己牺牲了多少。这时候女主角注意到了猫猫的牺牲,不同于一周目的男主角,女主角就算是放弃对她很重要的冠军也不希望猫猫受伤。受到感动的主角突然发现了第三种选择,自己实际上只是寄存在猫猫的身体之中,如果用自己的神念代替猫猫承受【某种伤害】,这样女主角和猫猫就都不用受伤了。猫猫的身体之中会失去主角的意识,变回普通的猫,女主角也可以拿到冠军。主角决定只牺牲自己,帮助一周目帮他良多的猫猫和愿意放弃冠军的女主角,而这正好达成了【某种条件】,进入了happy end。</summary></details>

</blockquote></details>

</blockquote></details>
<details><summary>尾声:男主角在事件后重新回到了自己的肉身。他被第二周目的女主角的自我牺牲所感动,对女主角暗生情愫,并且因为和女主角生活过一段时间,所以对她格外了解。女主角并不知道男主角变成过猫,但是通过蛛丝马迹她开始有所怀疑。</summary><blockquote>
</blockquote></details>

% 开头:男主在路上路遇燕火,燕火提出自己从未离开过学校,想要看一看外面的样子。男主内心觉得不对,但还是顺口答应了下来。
几个场景:
小西门的人脸识别,男主还在因为花式识别不上而摆姿势,回头一看燕火不知道什么时候已经出去了。男主仔细回想有没有听到“认证成功”的语音,但怎么也想不起来。
	南门的各种店铺:男主一一介绍。一路走到了南门才想起来燕火一路上都没搭过话(需要演出配合,使得之前的男主独白不显得突兀)。回头一看却发现燕火就在自己背后。
结尾:回到学校之后,男主开始仔细回想各种细节,觉得有些不对,燕火到底是……什么人?燕元出现,对男主说还是不要细想为好,男主问道你到底是谁?燕元笑道,我当然是燕元啊。

% \hypertarget{fengjirichang}{}
事件开始:风纪委员在某次关门收拾的时候发现少了一把椅子。恰好男主在活动室整理资料出来晚了,于是被指为犯人。男主辩称自己是无辜的,被要求说如果你是无辜的就自己找出犯人,男主无奈开始查案。
解决:男主一番查案无果,怀疑是灵异所为,但风纪委员不相信灵异。男主一天对风纪委员说你看锁门之后也是会丢的,不信我们来试一试,风纪委员应约,发现不仅东西丢了,还遇到许多其他吓人现象,被吓的不轻,也只好相信。(附加设定:之后会怕鬼)
幕后:男主在查案无果后只好选择向主角团求助,众人合伙上演了一波灵异事件。风纪委员所见其实是其他人所为。(系统:可以增加一些躲避小游戏?)最后其他人被关在新太阳里面一晚上,在活动室彻夜聊天。
结尾:众人问道椅子究竟去哪儿了。男主直言不知道。众人调侃着一定是你弄坏了,记得买把新的放回去一类,男主说真的不是我。事件结束,风机委员一边担心着闹鬼一边锁门。在门背后阴影中,似乎有什么东西拖着椅子消失在黑暗中……

% \section{男主角事件大纲}
\hypertarget{event:nzj}{}
第一阶段:主角团集合
	序盘部分。需要安排与舍友女主妹妹程序员4人的事件。
	大致按照顺序,初步构想如下:
1.	舍友。一开始带领男主参观校园,路过【某些地点】时,向男主讲灵异故事。男主不以为意,但是被舍友强塞了很多【相关资料】(可以是前人遗留物)。
a)	某些地点:可以是后面真正出现灵异的地点,但是灵异故事可以和后面真正发生的不同,最好越可疑越不可能越好。
b)	相关资料:可以是前人的遗留物,暗示校园里从前可能也有灵异事件。保持神秘感。
c)	舍友的身份:此时不涉及校庆,舍友以单纯的游手好闲型角色登场。会长的身份将暂时保留。
2.	女主。以经典的转角碰撞作为认识的契机,因为看到男主手里的“相关资料”很感兴趣,追着男主问东问西。男主一开始有些不知所措,渐渐地被热情所感染,也开始研究其资料来。女主提出要和舍友见一下。男主联系舍友之后,舍友当下决定成立一个社团。
a)	串联:可以在提出决定成立社团之后,想到妹妹,串联到妹妹事件,以回忆的形式插叙。
3.	妹妹。以之前我写过的小事件作为引入(插叙回忆)。舍友邀请妹妹加入社团。
a)	邀请加入:可以有半骗半引诱的感觉,凸显妹妹的天然,打下舍友逗妹妹玩的人物关系基调。也可以是单纯的妹妹对这个很感兴趣,凸显只有男主一人对灵异不太上心,衬托后文的转变。
4.	程序员。社团需要(情报来源/技术支持),需要一个技术人才。程序员因为(是论坛/树洞的管理员/在通选课上表现出惊人的能力)被注意到,顺势邀请加入。加入的动机倾向于否定灵异事件,以科学的方式给出解释。‘
a)	否定:否定是另一种程度强烈的关心。另外对于灵异的不同立场可以引发激烈的讨论,用于体现男主难以加入话题/难以融入的气氛。
5.	爷爷。和爷爷打电话作为一个【一周一次?】的活动,标示一个小段落。电话内容是听爷爷讲一些过去的往事,兼调节故事节奏用。故事大概有以下几个:
a)	上学时会被老教授拉着出去,在校园里散步,讲过去的故事,一讲就是半天
b)	未名湖当时环境不是很好,有时候会去湖边捡垃圾。
c)	石舫在夜晚会动。
d)	(其他闲聊)
6.	路人。路人的聊天会影响男主角的内心。路人设置在地图上几个必经之处。路人的一些闲聊如下:
a)	两个研究生谈论专业课上不靠谱的队友。“这人本科不知道哪儿的,根本算不上北大的啊”。
b)	两个本科生谈论不靠谱的老师。“这老师在北大混了这么多年,就一直教个基础课。也没见他拿出什么科研成果,就天天混日子,还经常发微博以北大老师自居,他也配?”
c)	其他碎碎念。缓解紧张气氛。上面两个必然刷新,这里可以随机刷新一些。
第二部分:事件发生
1.	引入:众人获得一些情报(程序员的情报渠道/众人独立来源分别听说),石舫在夜晚会动,众人决定前去调查。
2.	事件:众人夜晚来到此处,发现确有其事,众人对此反应不一。只有男主看不到这一现象,面对热情高涨的众人不禁略微感到尴尬。众人走上石舫,大家都小心翼翼,但男主并不觉得石舫会动。夜晚的石舫和水面的边界模糊,水面随风泛起微微的波纹,男主不禁觉得内心有点动摇……就在此时不慎掉入水中。男主狼狈爬上岸。
回去?路上问了一些路人(地图交互),有的说确有其事,有的嗤之以鼻。
3.	思考:晚上男主正好想到给爷爷打个电话。爷爷像往常一样说着故事。平常男主会悠闲地听着,但是这次不禁有些焦躁。回想起白天听到路人的碎碎念,便唐突的问起爷爷,什么才算一个北大人。男主听到回答之后,陷入沉思,缓缓睡去。
a)	【爷爷回答】:你觉得你是,你在乎的人觉得你是,你当然就是。
4.	独自思考:众人决定去吃个夜宵,顺便商讨事件。男主随着去了,但加入不了讨论,随声附和让他感觉越来越尴尬,因此借口身体不适离开。 之后的展开可能有数种:
a)	伙伴们的鼓励。女主察觉到了男主的情结,鼓动大家去和男主聊天。最终男主解开心结。仅女主一人也可。
b)	偶然的推动。路过的路人游客问路,激起了男主对于北大的归属感。
c)	家人。可以结合北大爷爷的设定,进行一些讨论。
d)	自己的反思。心理系学生对自我的精神分析。可以结合b) c)。
5.	会长说校园其他各处可能还有类似的事情,于是拉着大家去校园各处乱逛。众人参照前文【资料】找寻各处。
a)	【资料】:类似图鉴一样,可以让玩家随时在游戏中查看。
b)	各处发生的事情:发生和爷爷类似的事情。男主感同身受,觉得自己无疑是属于这里的一份子。
i.	(遇到散步的老教工被拉着听故事)。走着走着回头一看大家都溜了之类的王道展开。
ii.	遇到路人不小心掉了塑料瓶在未名湖里。男主不假思索地探身把塑料瓶抓了回来(可以在地图上用男主的强制行动、q版人物变换表现)。
iii.	遇到之前路人闲聊的反转。“没想到这人干起某某活还挺厉害的”。“原来这个教授有两把刷子”。结合上面两个事件,逐渐衬托出男主渐强的归属感。
第三部分:事件解决
1.	男主回归团队,提出再去灵异地点考察一次。到了石舫,男主提出“为什么我们一定要解决这个问题呢?就让石舫在那里,有人觉得它会动,有人觉得不会,又有什么不可以的呢?”同时心想“就像有的人认为自己属于这里……而有的人不这么认为。”
2.	众人提出“再有人像你一样掉到水里怎么办?”男主微微一笑,胸中突然感到一股自信,这次自己不会再掉水里了、于是快速地跳上石舫,沿着边跑了一圈。众人微微惊叹。男主最后说道“在意这件事的人,自己自然会注意;不在意这件事的人,可能总也不会踏上石舫吧。”,并提议众人在石舫上坐下赏月。众人相应,在石舫上聊天欢笑彻夜。

总时间线
第一天	舍友早上提议带领男主参观校园。男主接受之后二人在校园浏览。路过【某些地点】(包括石舫)时,舍友讲述灵异故事。一圈游览下来,舍友把【资料】强塞给男主。晚上发生爷爷对话1.
第二天	男主前去上课,因为路线不熟悉差点迟到。转角正好撞到女主。男主忙于去上课,没有注意自己掉了东西。女主捡到了【资料】。男主冲去上课。课程(可以就是公园大纲中提到的课程)留了一个小作业,男主见识到了坐在旁边的程序员用超高技术力瞬间写完。 下课之后出门,恰巧遇到女主。男主这才反应过来自己掉了东西。女主表示看过这份资料,觉得很有意思。男主一开始不以为意,但是在热情的解说下渐渐提起了兴趣。女主问道这资料这么详细,相比你做了很多研究吧?男主拖出资料来源于舍友,女主顺势要求见一下舍友。舍友这时突然出现(“我全都听到了!”),提议成立灵异部。程序员在一旁注意到了此事,但是保持了沉默。
第三天	我之前写过的小事件。舍友顺势邀请妹妹加入社团。晚上,爷爷对话2
第四天	众人在活动室(新太阳?)集合。室友提出要搜集目击情报。男主想到课上令人印象深刻的程序员。下一节课的时候男主鼓起勇气向程序员提出此事,程序员令人意外地一口答应,理由是想要否定灵异事件。
第五天	事件发生。程序员在论坛、树洞上发现石舫传言。女主在系图书馆查到了相关资料。妹妹说她亲眼看到。舍友说他听朋友的朋友的朋友说的(不禁让人怀疑真实性)。总而言之,众人决定前去调查。调查途中听到路人关于北大人的碎碎念。晚上调查,男主落水。爷爷对话3.
第六天	舍友提议的乱逛。可以设计一些反应主角团其他几人性格的事件(未确定)。遇到和爷爷相似的时间,提升了自己的归属感。遇到路人碎碎念的反转。
第七天	重回石舫,向众人展示自信。最后给爷爷打电话,向他讲述整个故事。爷爷听后笑道看来你在北大也有了自己的生活,以后少给我打电话,多陪朋友们吧。主角也恍然意识到,主角团众人早已是朋友。主角向爷爷的点拨道谢,并表示就算是偶尔也一定会再打电话来的。(可以在后续事件遇到瓶颈的时候再搬出爷爷来,待定。)

补充:
注1:第一夜,舍友带领男主参观校园。(小时候,男主曾经来过北大,对于湖光塔影以及石舫上的少女,有着模糊的印象)未名湖边,眼前景色与童年回忆在脑中交织,心头百感交集。男主将目光投向石舫,月光倾注其上,宛若回忆中的光景。突然,男主听到了室友的惊呼——石舫上仿佛凭空多了一座建筑。定睛瞧去,却并无发现。待室友回过神来,那座建筑已经消失了。
此后,室友发现灵异,同时察觉到校内的流言。在校庆筹办会议上反馈此事,X教授联想到四十年前的事件,给予「资料」要求室友调查此事。室友邀请男主来帮忙。/X教授注意到男主入学,知道男主是老同学的后人,因此钦定男主来调查,要求室友把「资料」给男主。
注2:四十年前事件的资料,以此为驱动力,在学校巡礼,逐渐解锁地图。
起初,男主不知道这些资料是“四十年前的”,“校庆的”,仅当作“校园怪谈”。
第一夜发现石舫的灵异后,学校内也出现了有关其他灵异的传闻,这吸引男主进一步调查。
男主一开始抱着“找乐子”的态度,到资料中的地点调查,仅作为一种茶余饭后的消遣。(深层原因:想要见证灵异,从而证明自己融入北大,在几次扑空后这一冲动增强)
男主唯一能察觉的灵异,是燕元燕火。
注3:男主在调查过程中邂逅了女主(“她仿佛在跟踪我”,总是出现在相同的地方)/男主救了女主(某个灵异导致女主遇到困境,要仔细设计,落水什么的太平凡了;不一定是拯救性命,参考冰菓,把女主从上锁的房间捞出来也是一种救)
这样更美一些,傍晚的樱花树、入夜的未名湖、清晨的静园草坪,男主凭借好奇心调查神奇之事,于是遇到了女主。(参考Clannad,前天是小兔子,昨天是小鹿,今天是你)在课堂的相遇带有教学楼的拥挤和嘈杂感,不够美。
舞台的选取:北大足够丰富,可以选一些有异于其他校园作的场景,这也是我们的特色。
女主为什么会在?民俗学作业。民俗学是X教授讲的。邀请男主选民俗学。后续猫事件围绕民俗学展开。成了。

新增:在此之前
男主的两段历史:
北大梦:祖父故事,中学努力,高考失利
北大行:夏令营,独自出走,夜游北大,月下少女
前一个是开头就可以交代的,后一个是在校园生活中逐渐发挥作用的。当男主迈入北大时,应该是何种心情呢?(要刻画这一个瞬间,表现思路类似于clannad灰色-彩色,但具体手法要换一换)“我选择从西门进入北大”“灰白的世界,透明的面孔”


时间轴:
D1夜,未名湖,唤起回忆,初见灵异。
D2 早期剧情双线并行:灵异线,室友提供资料,男主调查。校庆线,发现有人在筹办校庆,体验现充活动。
D3 遇女主,解决小困难小挑战,类似于冰菓开头的事件(待补充)。与女主交流情报,与女主约课民俗学,认识了程序大佬(来证伪的)。同时在论坛中发现了网络巨人。
D4 校庆线,调查室友行踪(花花公子)的一天,解锁妹妹,暗示室友和校庆的关系。顺着女主遇到的小困难,需要程序大佬的加入。队伍组齐,夜探未名湖,再次验证了灵异(只有男主没看见)。
祖先托梦等,解决灵异,待补充……
最终 解决第一个灵异。揭示室友真实身份,校庆线与灵异线汇为一条。社团(部门)正式成立。在石舫上,扣舷而歌之,不知东方之既白。

% \include{素材/程序员事件/程序员事件大纲}
% 背景设定
	会长曾经作为足球队的主力,在一场关键比赛中失利。
故事开始
	(会长单人视角,类似梦境)会长在收拾东西的时候翻出了一颗积满灰尘的足球。会长看着足球,用手擦了擦,似乎看见了上面写着的一些名字……不禁一时沉浸在回忆之中。
“当年,我也有过这样的经历啊……”就在此时,他仿佛听见一个声音:“是啊,那时候如此辉煌,现在却落得这么个无人问津的下场……”会长环顾四周,却不见人影。“那时候你不是很后悔吗?你一定也这样想过吧:早知如此,还不如从一开始就不去踢什么足球……”“不!我……”“如果你不这么想的话,就证明给我看!要我说,足球这种东西,还不如不存在!”
	第二天,会长去梦里那个找到的地方翻找,果然找到了球。会长的翻找引来了众人的好奇,众人纷纷询问这是什么。会长答道这是一颗足球,众人却惊讶的表示“足球是什么?”会长嘲笑了一番几个人,提议说不如我们几个来玩,于是教给众人规则,众人愉快的边学边踢球。快乐的运动时间之后,会长闲聊时提到曾经还有北大杯的足球比赛,众人却表示北大杯有篮球排球等各个项目,唯独没听说过足球。会长笑道“这不可能”,掏出手机想要找出足球队的战报推送等……却发现遍寻不见。会长慌忙地问了几个路人,却没人听说过足球。(地图互动)这时他回想起昨天的梦境,恍然悟道一切是足球从中作梗。
	当天晚上,会长再次梦见足球。会长在梦中质问足球是不是它搞的鬼,足球答道想要结束“足球并不存在”的情况,就组织起来一场比赛来证明足球是有意义的。会长抱怨道这种情况怎么可能凑齐11人的阵容,足球便说那五人比赛也行,谅你也凑不起来。会长在纠结中缓缓意识淡去。
	
故事中盘
	会长第二天来到活动室正想提出“组织一场比赛”,女主却抢先说道“有一只队伍想要踢比赛,找到我们问我们能不能接。”其他三人惊讶。会长顺势高兴地接下,提出要主角团进行足球特训。
	中略,标准热血体育竞技题材剧情
	到了正式比赛的那天,会长来到球场上,惊讶的发现对方队伍正是自己曾经足球队的队友。队友们似乎也忘记了曾经踢球的时光,大家一起在赛场上享受到了最开始踢球的乐趣。比赛最终可以有(类似clannad的)传给队友/自己射门的选项,影响比赛结局(可以不影响后续剧情,或者是不做BE(待定))
	比赛结束之后,会长躺在草坪上享受着余韵。突然他感觉到不对,一切进行的过于顺利了,自己刚在梦中听到要组织比赛就正好有一支队伍,队伍成员还碰巧都是以前足球队的……一切显得过于巧合。就在这时主角团出现,带着水/运动饮料(CG,会长躺在地上的视角,周围一圈是主角团,正上方是刺眼的阳光),说道果然最后你还是注意到了,进入主角团视角的回忆。

主角团视角,部分插入顺序叙事,部分作为保留在最后再托出(红色)
	主角团在第一天踢完球之后就有人感受到了不对劲(暂定女主)。当天召集了除了会长以外的其他人,众人觉得可能是神念从中作梗。在程序员的搜索下,发现足球确实是存在的,只不过在北大的范围内大家都仿佛集体失忆一般。众人不知道解决的办法,一筹莫展。此时女主提出某事,众人一同谋划。某事就是策划比赛,女主提出从未见过会长这么“真正的高兴”,他一定是真心喜欢足球,众人应允。天色已晚,众人决定具体细节择日再议。
	第二天女主突然提出有队,众人惊讶(内心:这你没说过啊!)事后女主说道突然想到如果会长所言不假,那么他以前在足球队一定有队友,我们不如把他们找出来,组织一场比赛。众人感觉又被牵着走了,但是大家也仍然积极响应。众人为了找到以前的球队成员费尽周折(比如在院系课程下课时在门口踢球观察众人反应,在课程群发足球图片等,可以被赶出教学楼/被提出群聊等)。最终终于找到了以前的球队成员(这部分可以省去一些细节,略写。)
	可以描写一些主角团众人因为双线而感到疲惫/险些露马脚的线索,放在前面作为伏笔线索。

结尾
	会长无奈地笑道真是服了你们了,内心却因为众人而微微有些感动(Cg模糊处理)。不知何时,球场周围人渐渐多了起来,是平常就来踢球的球友们来催促众人“不踢了就赶紧让场子”。会长从躺着一跃而起(镜头旋转起身),“谁说我不踢了?今天我就要踢个痛快!”
Misc:事件解决后,周围路人的碎碎念也开始提及足球。
第二天,会长问众人道:“那你们是怎么知道组织比赛就能结束神念的呢。”众人表示不知。会长奇怪道那你们为什么要费劲组织比赛呢?众人答道“看你的表情,大家都知道你期盼着再继续踢球啊。”会长意识到无关神念,大家也一直关注着自己。
(接下来可以在整个猛男落泪,不过可能有点矫情,待定)

% 如果男主角到了某个badend(或者没能成功地完成某个事件),出现燕火给出一些提示;如果男主角顺利地完成了某个事件,那么就遇到燕元,和燕元对话。

在某个灵异事件里,主角小队需要解决一个迷题;这时候玩家可以选择任何一个主角小队里的角色当做主角色操作,而这个角色解决迷题的方式和遇到的剧情是固定的;这时候玩家可以选择任何一个主角小队里的角色当做主角色操作,而这个角色解决迷题的方式和遇到的剧情是固定的;玩家先操作主角团的a,a如果没有能解决迷题,那么玩家可以选择操作主角团的b;这是有时间继承顺序的;比如有一个只有a能发现的秘密,操作完a之后,再操作b可以根据a获得的秘密找到迷题的答案;如果先操作b再操作a的话可能会死锁;玩家操作a的时候,主角团的bcd可以像幽灵一样绕着a,时不时发出吐槽但是没法行动;一个是不选择让玩家进行行为上的选项,而改成选角色顺序,角色的行为是固定的但是玩家可以选择这一套套不同的行为方案;另一个是某个主角进行行为时,其他人无法干涉但是可以欢乐吐槽,并且可以轮换,这样可以实现你上你也行.

最后的校庆类似于幻想嘉年华。


%  煮熟的生蚝还是不是生蚝?
 
 如果一个人自杀了,那么这个世界上是多了一个自杀的人,还是少了一个自杀的人?
 
 人与人之间是平等的,但是有些有钱人,总是要比其他人更加平等一点。
 
 从前有个人从来都不洗脸,大家都叫他脏脸;有一个人嘴总是合不拢,大家都叫他歪嘴;还有一个人总是随地大小便,大家都叫他不要这么做。
 
 有一说一,这件事大家懂得都懂,不懂得,说了你也不明白,不如不说。你们也别来问我怎么了,利益牵扯太大,说了对你们也没什么好处,当不知道就行了,其余的我只能说这里面水很深,牵扯到很多大人物。详细资料你们自己找是很难找的,网上大部分已经删除干净了,所以我只能说懂得都懂,不懂得也没办法。
 
 回复TD退订
 
 啤酒饮料矿泉水、花生瓜子八宝粥,前面的旅客麻烦让一让。
 
 后浪
 
 (ちょっと待って),这样子讲话有什么错吗?呐(ねえ),告诉我啊。搜噶(そっか),你们已经不喜欢了啊……真是冷酷的人呢,果咩纳塞(ごめんなさい),让你看到不愉快的东西了。像我这样的人,果然消失就好了呢。也许只有在二次元的世界里,才有真正的美好存在的吧,呐(ねえ)?\url{https://zh.moegirl.org.cn/index.php?title=%E5%91%90}
 
 我五分钟之后到。如果没到就再把上句话再念一遍。
 
 人生有梦,各自精彩。
 
 如果你对自己的外表不自信,请记住,和你相貌相近的每个祖先都找到了愿意和他结合的人。当然还有更多的旁系亲戚没找到。
 
 扒马褂
 
 



\end{document}
