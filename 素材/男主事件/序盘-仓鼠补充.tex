补充:
注1:第一夜,舍友带领男主参观校园。(小时候,男主曾经来过北大,对于湖光塔影以及石舫上的少女,有着模糊的印象)未名湖边,眼前景色与童年回忆在脑中交织,心头百感交集。男主将目光投向石舫,月光倾注其上,宛若回忆中的光景。突然,男主听到了室友的惊呼——石舫上仿佛凭空多了一座建筑。定睛瞧去,却并无发现。待室友回过神来,那座建筑已经消失了。
此后,室友发现灵异,同时察觉到校内的流言。在校庆筹办会议上反馈此事,X教授联想到四十年前的事件,给予「资料」要求室友调查此事。室友邀请男主来帮忙。/X教授注意到男主入学,知道男主是老同学的后人,因此钦定男主来调查,要求室友把「资料」给男主。
注2:四十年前事件的资料,以此为驱动力,在学校巡礼,逐渐解锁地图。
起初,男主不知道这些资料是“四十年前的”,“校庆的”,仅当作“校园怪谈”。
第一夜发现石舫的灵异后,学校内也出现了有关其他灵异的传闻,这吸引男主进一步调查。
男主一开始抱着“找乐子”的态度,到资料中的地点调查,仅作为一种茶余饭后的消遣。(深层原因:想要见证灵异,从而证明自己融入北大,在几次扑空后这一冲动增强)
男主唯一能察觉的灵异,是燕元燕火。
注3:男主在调查过程中邂逅了女主(“她仿佛在跟踪我”,总是出现在相同的地方)/男主救了女主(某个灵异导致女主遇到困境,要仔细设计,落水什么的太平凡了;不一定是拯救性命,参考冰菓,把女主从上锁的房间捞出来也是一种救)
这样更美一些,傍晚的樱花树、入夜的未名湖、清晨的静园草坪,男主凭借好奇心调查神奇之事,于是遇到了女主。(参考Clannad,前天是小兔子,昨天是小鹿,今天是你)在课堂的相遇带有教学楼的拥挤和嘈杂感,不够美。
舞台的选取:北大足够丰富,可以选一些有异于其他校园作的场景,这也是我们的特色。
女主为什么会在?民俗学作业。民俗学是X教授讲的。邀请男主选民俗学。后续猫事件围绕民俗学展开。成了。

新增:在此之前
男主的两段历史:
北大梦:祖父故事,中学努力,高考失利
北大行:夏令营,独自出走,夜游北大,月下少女
前一个是开头就可以交代的,后一个是在校园生活中逐渐发挥作用的。当男主迈入北大时,应该是何种心情呢?(要刻画这一个瞬间,表现思路类似于clannad灰色-彩色,但具体手法要换一换)“我选择从西门进入北大”“灰白的世界,透明的面孔”


时间轴:
D1夜,未名湖,唤起回忆,初见灵异。
D2 早期剧情双线并行:灵异线,室友提供资料,男主调查。校庆线,发现有人在筹办校庆,体验现充活动。
D3 遇女主,解决小困难小挑战,类似于冰菓开头的事件(待补充)。与女主交流情报,与女主约课民俗学,认识了程序大佬(来证伪的)。同时在论坛中发现了网络巨人。
D4 校庆线,调查室友行踪(花花公子)的一天,解锁妹妹,暗示室友和校庆的关系。顺着女主遇到的小困难,需要程序大佬的加入。队伍组齐,夜探未名湖,再次验证了灵异(只有男主没看见)。
祖先托梦等,解决灵异,待补充……
最终 解决第一个灵异。揭示室友真实身份,校庆线与灵异线汇为一条。社团(部门)正式成立。在石舫上,扣舷而歌之,不知东方之既白。