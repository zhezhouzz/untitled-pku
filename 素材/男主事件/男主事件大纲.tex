\section{男主角事件大纲}
\hypertarget{event:nzj}{}
第一阶段:主角团集合
	序盘部分。需要安排与舍友女主妹妹程序员4人的事件。
	大致按照顺序,初步构想如下:
1.	舍友。一开始带领男主参观校园,路过【某些地点】时,向男主讲灵异故事。男主不以为意,但是被舍友强塞了很多【相关资料】(可以是前人遗留物)。
a)	某些地点:可以是后面真正出现灵异的地点,但是灵异故事可以和后面真正发生的不同,最好越可疑越不可能越好。
b)	相关资料:可以是前人的遗留物,暗示校园里从前可能也有灵异事件。保持神秘感。
c)	舍友的身份:此时不涉及校庆,舍友以单纯的游手好闲型角色登场。会长的身份将暂时保留。
2.	女主。以经典的转角碰撞作为认识的契机,因为看到男主手里的“相关资料”很感兴趣,追着男主问东问西。男主一开始有些不知所措,渐渐地被热情所感染,也开始研究其资料来。女主提出要和舍友见一下。男主联系舍友之后,舍友当下决定成立一个社团。
a)	串联:可以在提出决定成立社团之后,想到妹妹,串联到妹妹事件,以回忆的形式插叙。
3.	妹妹。以之前我写过的小事件作为引入(插叙回忆)。舍友邀请妹妹加入社团。
a)	邀请加入:可以有半骗半引诱的感觉,凸显妹妹的天然,打下舍友逗妹妹玩的人物关系基调。也可以是单纯的妹妹对这个很感兴趣,凸显只有男主一人对灵异不太上心,衬托后文的转变。
4.	程序员。社团需要(情报来源/技术支持),需要一个技术人才。程序员因为(是论坛/树洞的管理员/在通选课上表现出惊人的能力)被注意到,顺势邀请加入。加入的动机倾向于否定灵异事件,以科学的方式给出解释。‘
a)	否定:否定是另一种程度强烈的关心。另外对于灵异的不同立场可以引发激烈的讨论,用于体现男主难以加入话题/难以融入的气氛。
5.	爷爷。和爷爷打电话作为一个【一周一次?】的活动,标示一个小段落。电话内容是听爷爷讲一些过去的往事,兼调节故事节奏用。故事大概有以下几个:
a)	上学时会被老教授拉着出去,在校园里散步,讲过去的故事,一讲就是半天
b)	未名湖当时环境不是很好,有时候会去湖边捡垃圾。
c)	石舫在夜晚会动。
d)	(其他闲聊)
6.	路人。路人的聊天会影响男主角的内心。路人设置在地图上几个必经之处。路人的一些闲聊如下:
a)	两个研究生谈论专业课上不靠谱的队友。“这人本科不知道哪儿的,根本算不上北大的啊”。
b)	两个本科生谈论不靠谱的老师。“这老师在北大混了这么多年,就一直教个基础课。也没见他拿出什么科研成果,就天天混日子,还经常发微博以北大老师自居,他也配?”
c)	其他碎碎念。缓解紧张气氛。上面两个必然刷新,这里可以随机刷新一些。
第二部分:事件发生
1.	引入:众人获得一些情报(程序员的情报渠道/众人独立来源分别听说),石舫在夜晚会动,众人决定前去调查。
2.	事件:众人夜晚来到此处,发现确有其事,众人对此反应不一。只有男主看不到这一现象,面对热情高涨的众人不禁略微感到尴尬。众人走上石舫,大家都小心翼翼,但男主并不觉得石舫会动。夜晚的石舫和水面的边界模糊,水面随风泛起微微的波纹,男主不禁觉得内心有点动摇……就在此时不慎掉入水中。男主狼狈爬上岸。
回去?路上问了一些路人(地图交互),有的说确有其事,有的嗤之以鼻。
3.	思考:晚上男主正好想到给爷爷打个电话。爷爷像往常一样说着故事。平常男主会悠闲地听着,但是这次不禁有些焦躁。回想起白天听到路人的碎碎念,便唐突的问起爷爷,什么才算一个北大人。男主听到回答之后,陷入沉思,缓缓睡去。
a)	【爷爷回答】:你觉得你是,你在乎的人觉得你是,你当然就是。
4.	独自思考:众人决定去吃个夜宵,顺便商讨事件。男主随着去了,但加入不了讨论,随声附和让他感觉越来越尴尬,因此借口身体不适离开。 之后的展开可能有数种:
a)	伙伴们的鼓励。女主察觉到了男主的情结,鼓动大家去和男主聊天。最终男主解开心结。仅女主一人也可。
b)	偶然的推动。路过的路人游客问路,激起了男主对于北大的归属感。
c)	家人。可以结合北大爷爷的设定,进行一些讨论。
d)	自己的反思。心理系学生对自我的精神分析。可以结合b) c)。
5.	会长说校园其他各处可能还有类似的事情,于是拉着大家去校园各处乱逛。众人参照前文【资料】找寻各处。
a)	【资料】:类似图鉴一样,可以让玩家随时在游戏中查看。
b)	各处发生的事情:发生和爷爷类似的事情。男主感同身受,觉得自己无疑是属于这里的一份子。
i.	(遇到散步的老教工被拉着听故事)。走着走着回头一看大家都溜了之类的王道展开。
ii.	遇到路人不小心掉了塑料瓶在未名湖里。男主不假思索地探身把塑料瓶抓了回来(可以在地图上用男主的强制行动、q版人物变换表现)。
iii.	遇到之前路人闲聊的反转。“没想到这人干起某某活还挺厉害的”。“原来这个教授有两把刷子”。结合上面两个事件,逐渐衬托出男主渐强的归属感。
第三部分:事件解决
1.	男主回归团队,提出再去灵异地点考察一次。到了石舫,男主提出“为什么我们一定要解决这个问题呢?就让石舫在那里,有人觉得它会动,有人觉得不会,又有什么不可以的呢?”同时心想“就像有的人认为自己属于这里……而有的人不这么认为。”
2.	众人提出“再有人像你一样掉到水里怎么办?”男主微微一笑,胸中突然感到一股自信,这次自己不会再掉水里了、于是快速地跳上石舫,沿着边跑了一圈。众人微微惊叹。男主最后说道“在意这件事的人,自己自然会注意;不在意这件事的人,可能总也不会踏上石舫吧。”,并提议众人在石舫上坐下赏月。众人相应,在石舫上聊天欢笑彻夜。

总时间线
第一天	舍友早上提议带领男主参观校园。男主接受之后二人在校园浏览。路过【某些地点】(包括石舫)时,舍友讲述灵异故事。一圈游览下来,舍友把【资料】强塞给男主。晚上发生爷爷对话1.
第二天	男主前去上课,因为路线不熟悉差点迟到。转角正好撞到女主。男主忙于去上课,没有注意自己掉了东西。女主捡到了【资料】。男主冲去上课。课程(可以就是公园大纲中提到的课程)留了一个小作业,男主见识到了坐在旁边的程序员用超高技术力瞬间写完。 下课之后出门,恰巧遇到女主。男主这才反应过来自己掉了东西。女主表示看过这份资料,觉得很有意思。男主一开始不以为意,但是在热情的解说下渐渐提起了兴趣。女主问道这资料这么详细,相比你做了很多研究吧?男主拖出资料来源于舍友,女主顺势要求见一下舍友。舍友这时突然出现(“我全都听到了!”),提议成立灵异部。程序员在一旁注意到了此事,但是保持了沉默。
第三天	我之前写过的小事件。舍友顺势邀请妹妹加入社团。晚上,爷爷对话2
第四天	众人在活动室(新太阳?)集合。室友提出要搜集目击情报。男主想到课上令人印象深刻的程序员。下一节课的时候男主鼓起勇气向程序员提出此事,程序员令人意外地一口答应,理由是想要否定灵异事件。
第五天	事件发生。程序员在论坛、树洞上发现石舫传言。女主在系图书馆查到了相关资料。妹妹说她亲眼看到。舍友说他听朋友的朋友的朋友说的(不禁让人怀疑真实性)。总而言之,众人决定前去调查。调查途中听到路人关于北大人的碎碎念。晚上调查,男主落水。爷爷对话3.
第六天	舍友提议的乱逛。可以设计一些反应主角团其他几人性格的事件(未确定)。遇到和爷爷相似的时间,提升了自己的归属感。遇到路人碎碎念的反转。
第七天	重回石舫,向众人展示自信。最后给爷爷打电话,向他讲述整个故事。爷爷听后笑道看来你在北大也有了自己的生活,以后少给我打电话,多陪朋友们吧。主角也恍然意识到,主角团众人早已是朋友。主角向爷爷的点拨道谢,并表示就算是偶尔也一定会再打电话来的。(可以在后续事件遇到瓶颈的时候再搬出爷爷来,待定。)

补充:
注1:第一夜,舍友带领男主参观校园。(小时候,男主曾经来过北大,对于湖光塔影以及石舫上的少女,有着模糊的印象)未名湖边,眼前景色与童年回忆在脑中交织,心头百感交集。男主将目光投向石舫,月光倾注其上,宛若回忆中的光景。突然,男主听到了室友的惊呼——石舫上仿佛凭空多了一座建筑。定睛瞧去,却并无发现。待室友回过神来,那座建筑已经消失了。
此后,室友发现灵异,同时察觉到校内的流言。在校庆筹办会议上反馈此事,X教授联想到四十年前的事件,给予「资料」要求室友调查此事。室友邀请男主来帮忙。/X教授注意到男主入学,知道男主是老同学的后人,因此钦定男主来调查,要求室友把「资料」给男主。
注2:四十年前事件的资料,以此为驱动力,在学校巡礼,逐渐解锁地图。
起初,男主不知道这些资料是“四十年前的”,“校庆的”,仅当作“校园怪谈”。
第一夜发现石舫的灵异后,学校内也出现了有关其他灵异的传闻,这吸引男主进一步调查。
男主一开始抱着“找乐子”的态度,到资料中的地点调查,仅作为一种茶余饭后的消遣。(深层原因:想要见证灵异,从而证明自己融入北大,在几次扑空后这一冲动增强)
男主唯一能察觉的灵异,是燕元燕火。
注3:男主在调查过程中邂逅了女主(“她仿佛在跟踪我”,总是出现在相同的地方)/男主救了女主(某个灵异导致女主遇到困境,要仔细设计,落水什么的太平凡了;不一定是拯救性命,参考冰菓,把女主从上锁的房间捞出来也是一种救)
这样更美一些,傍晚的樱花树、入夜的未名湖、清晨的静园草坪,男主凭借好奇心调查神奇之事,于是遇到了女主。(参考Clannad,前天是小兔子,昨天是小鹿,今天是你)在课堂的相遇带有教学楼的拥挤和嘈杂感,不够美。
舞台的选取:北大足够丰富,可以选一些有异于其他校园作的场景,这也是我们的特色。
女主为什么会在?民俗学作业。民俗学是X教授讲的。邀请男主选民俗学。后续猫事件围绕民俗学展开。成了。

新增:在此之前
男主的两段历史:
北大梦:祖父故事,中学努力,高考失利
北大行:夏令营,独自出走,夜游北大,月下少女
前一个是开头就可以交代的,后一个是在校园生活中逐渐发挥作用的。当男主迈入北大时,应该是何种心情呢?(要刻画这一个瞬间,表现思路类似于clannad灰色-彩色,但具体手法要换一换)“我选择从西门进入北大”“灰白的世界,透明的面孔”


时间轴:
D1夜,未名湖,唤起回忆,初见灵异。
D2 早期剧情双线并行:灵异线,室友提供资料,男主调查。校庆线,发现有人在筹办校庆,体验现充活动。
D3 遇女主,解决小困难小挑战,类似于冰菓开头的事件(待补充)。与女主交流情报,与女主约课民俗学,认识了程序大佬(来证伪的)。同时在论坛中发现了网络巨人。
D4 校庆线,调查室友行踪(花花公子)的一天,解锁妹妹,暗示室友和校庆的关系。顺着女主遇到的小困难,需要程序大佬的加入。队伍组齐,夜探未名湖,再次验证了灵异(只有男主没看见)。
祖先托梦等,解决灵异,待补充……
最终 解决第一个灵异。揭示室友真实身份,校庆线与灵异线汇为一条。社团(部门)正式成立。在石舫上,扣舷而歌之,不知东方之既白。
