\newcommand{\ruxue}[1]{
    \IfEqCase{#1}{
        {0}{
            \hyperlink{ruxue0}{【西门入校,遇燕元燕火,带到宿舍楼下,路上陈述一点】}
            \hypertarget{ruxue0}{}
            “北京大学有很多校门。我选择从西门入校。”
            “灰白的世界,透明的面孔。”
            男主是交换生,入学时不会看到一群人入学的盛况。这一点在后续“文化节”弥补。
        }
        {1}{
        【独自宿舍午睡,回想起一些情景——北大梦和北大行】
北大梦:祖父故事,中学努力,高考失利
北大行:夏令营,独自出走,夜游北大,月下少女
        }        
        {2}{
        【室友登场,带男主去校园北边逛,初见灵异】
小时候,男主曾经来过北大,对于湖光塔影以及石舫上的少女,有着模糊的印象。
未名湖边,眼前景色与童年回忆在脑中交织。
男主将目光投向石舫,突然听到了室友的惊呼——石舫上仿佛凭空多了一座建筑。待室友回过神来,那座建筑已经消失了。
        }        
        {3}{
        【首次上课,旁听民俗学,老师是X教授,课程作业要求搜集传说。女主和程序员露个脸。】
        【室友把资料交给男主】
        男主比较喜欢传说怪谈,所以旁听民俗学,没决定选。
        为什么男主获得了资料?几种解释:
        1、X教授钦定。在男主入学前,X教授注意此人是老同学的孙子。此前灵异已经发生,而男主的身份有利于其解决灵异,因此X教授安排会长成为男主室友,以此一步一步吸引男主投入事件。
        2、室友夜游时首次发现灵异,同时察觉到校内的流言。在校庆筹办会议上反馈此事,X教授联想到四十年前的事件,给予「资料」要求室友调查此事。正好男主民俗学课程与之有关,室友邀请男主帮忙。
        }
        {4}{
        【男主根据室友的资料,在校园中进行调查】
        四十年前事件的资料,以此为驱动力,在学校巡礼,逐渐解锁地图。起初,男主不知道这些资料是“四十年前的”,“校庆的”,仅当作“校园怪谈”。
	第一夜发现石舫的灵异后,学校内也出现了有关其他灵异的传闻,这吸引男主进一步调查。男主将这一活动合理化为“了解北大,完成民俗学作业”。(深层原因:想要见证灵异,从而证明自己融入北大,在几次扑空后这一冲动增强)
        }
        {5}{
        【某日黄昏,静园草坪的花树旁,邂逅女主。两人发现都在民俗学课堂上,男主决定选课。】
        女主和男主进行了类似的调查,“她仿佛在跟踪我”,总是出现在相同的地方。印象是:Clannad琴美“前天是小兔子,昨天是小鹿,今天是你”+樱之诗“樱花树下与凛重逢”
        也可以是男主救了女主:某个灵异导致女主遇到困境,要仔细设计,落水什么的太平凡了;不一定是拯救性命,参考冰菓,把女主从上锁的房间捞出来也是一种救。
        }
      
    }[\PackageError{tree}{Undefined option to tree: #1}{}]%
}

