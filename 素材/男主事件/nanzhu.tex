\newcommand{\nanzhu}[1]{
    \IfEqCase{#1}{
        {0}{
            男主角和女主角选课:【民俗学】。
        }
        {1}{
            舍友早上提议带领男主参观校园。男主接受之后二人在校园浏览。路过【某些地点】(包括石舫)时,舍友讲述灵异故事。一圈游览下来,舍友把【资料】强塞给男主。晚上发生爷爷对话1.
        }
        {2}{
            男主前去上课,因为路线不熟悉差点迟到。转角正好撞到女主。男主忙于去上课,没有注意自己掉了东西。女主捡到了【资料】。男主冲去上课。民俗学课程留了一个小作业,男主见识到了坐在旁边的程序员用超高技术力瞬间写完。 下课之后出门,恰巧遇到女主。男主这才反应过来自己掉了东西。女主表示看过这份资料,觉得很有意思。男主一开始不以为意,但是在热情的解说下渐渐提起了兴趣。女主问道这资料这么详细,相比你做了很多研究吧?男主拖出资料来源于舍友,女主顺势要求见一下舍友。舍友这时突然出现(“我全都听到了!”),提议成立灵异部。程序员在一旁注意到了此事,但是保持了沉默。
        }
        {3}{
            \hyperlink{meisheyou}{妹妹和舍友的小事件}。舍友顺势邀请妹妹加入社团。晚上,爷爷对话2
        }
        {4}{
            众人在活动室(新太阳?)集合。室友提出要搜集目击情报。男主想到课上令人印象深刻的程序员。下一节课的时候男主鼓起勇气向程序员提出此事,程序员令人意外地一口答应,理由是想要否定灵异事件。
        }
        {5}{
           事件发生。程序员在论坛、树洞上发现石舫传言。女主在系图书馆查到了相关资料。妹妹说她亲眼看到。舍友说他听朋友的朋友的朋友说的(不禁让人怀疑真实性)。总而言之,众人决定前去调查。可能发生之前写过的(问题比较大需要修改的)\hyperlink{huodongshi}{小事件}。调查途中听到路人关于北大人的碎碎念。晚上调查,男主落水。爷爷对话3.
        }
        {6}{
            舍友提议的乱逛。可以设计一些反应主角团其他几人性格的事件(未确定)。遇到和爷爷相似的时间,提升了自己的归属感。遇到路人碎碎念的反转。
        }
        {7}{
            重回石舫,向众人展示自信。最后给爷爷打电话,向他讲述整个故事。爷爷听后笑道看来你在北大也有了自己的生活,以后少给我打电话,多陪朋友们吧。主角也恍然意识到,主角团众人早已是朋友。主角向爷爷的点拨道谢,并表示就算是偶尔也一定会再打电话来的。(可以在后续事件遇到瓶颈的时候再搬出爷爷来,待定。)
        }
    }[\PackageError{tree}{Undefined option to tree: #1}{}]%
}