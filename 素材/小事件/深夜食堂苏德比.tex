\newcommand{\debi}[1]{
    \IfEqCase{#1}{
        {0}{
 深夜食堂
* 苏德比:CBD麻辣烫小哥,万事屋型角色,经常一边煮麻辣烫一边与同学攀谈,足不出户便能了解到校园各处发生的故事。【参考此间对一些食堂职工的报道】
* 分歧点:无。无论如何都是男主得到猫,仅作为引入事件。【可以改】
* 时间:序盘与男主事件之间,每天晚上。
        }
{1}{
【地点CBD,大地图,强制触发】
晚上,在麻辣烫摊上见到妹妹。
男主旁听麻辣烫小哥与其他同学聊天,发现关键词【石舫】。
男主感到震惊,试图询问,被小哥岔开话,“同学你是第一次来吗,先排队吧。“
【移动到菜架进行排队】小哥一边问“加不加辣”,一边和男主攀谈。
男主问到石舫有关的事情,小哥用很夸张的口气说“好像在闹鬼”。
谈话突然被打断。食品监督员(风纪委员)李星弥巡逻至此,向小哥问话。
问话结束后,男主的麻辣烫已经煮好了,小哥让男主“明天再聊”。
}


{2}{路过CBD,麻辣烫小哥未出现。但大地图可以有其他事件。}
{3}{再次遇到小哥,小哥称“遇到大事了,事情大到需要亲自出马。”两人交换情报,男主说了石舫事件,小哥透露“宿舍闹鬼”。RPG安乐椅推理,得出结论是猫叫。在宿舍地缝中找到猫。}

	* 再次来到麻辣烫摊位,对方称“你来晚了,石舫好像恢复正常了。”
	* 男主问前天为什么不在,小哥说“遇到大事了,事情大到需要我亲自出马。”
	* 男主追问细节,再次被要求点一份麻辣烫。
	* 男主称“石舫事件自己有参与,可以分享一些内幕”,小哥来了兴致,主动送男主一份麻辣烫,并告诉他稍事等候,收摊后细说。
	* 【AVG演出】两人围着CBD绕圈。小哥说,自己对校园生活很向往,但每天必须按时按点卖麻辣烫,只能通过和同学聊天,了解到各种故事。“我有夜宵,你有故事吗?”
	* 谈起了小哥第二天“”
	* 【推理小游戏,进入地图,人物行走图自动演出,男主和小哥画外音评论】
		* 在Day2的晚上,一群人走到摊位前。男生A告诉小哥,男生宿舍最近在闹鬼。
		* 其他人笑话他胆子小,“一群大汉住的地方,鬼都不去。”A很认真地说,这几天每到半夜两点,都会听到窗外有小孩的哭声。
		* 画外音男主推理——春天,可能是猫叫
		* 小哥说A也是这样猜的,但他去窗台上看,甚至下楼寻找,但并没有发现猫。
		* 小哥表示很好奇,问了其他人猫叫春的事情,有的人表示听说过(细节:思考过于认真,麻辣烫煮过头了)
		* 突然卖关子,说“有一个同学透露了重要的情报,自己也是因为这个情报才提前收摊的。”
	* 作为交换,男主讲起石舫的故事(简略表现)。讲完之后,小哥继续叙述。
	* 【推理小游戏,大地图】
		* 同学B说,自己住在A的同栋。他这几天感觉墙里有奇怪的声音。
		* 画外音男主推理——停暖气,水声
		* 小哥说B其实提到了,不只是水声,还包括抓挠爬骚的声音。
		* 男主组合线索,推理得到结论——前几天停暖气,宿舍旁工人修水管,打开了地下室。有一只猫不小心留在了里面,抓挠水管,发出叫声,导致了这种现象。
	* 小哥感到佩服。此时已经接近2点,他们突然听到了“哭声”。
	* 来到宿舍楼前,循声找去,声音果然从某个门内传来。
	* 小哥神奇地掏出了钥匙。“我在卖麻辣烫之前,其实是修水管的。”
	* 最终男主成功解救了猫。这只猫的身上很脏,但仍然很有精神。
* 后日谈
	* 无论猫事件触发与否,男主都会结识苏德比,此后也可以在这里打探情报。        

    }[\PackageError{tree}{Undefined option to tree: #1}{}]%
}