\newcommand{\jiaoshou}[1]{
    \IfEqCase{#1}{
    {0}{假面诗会
* X教授:女主的本科生导师,校庆筹办委员会顾问(隐藏身份)。中文系教授,文学社的指导老师,主要研究现代诗歌。
* 分歧点:如果触发本支线,男主会提前认识X教授,从而在妹妹事件中主动拜访,了解到妹妹所寻找的书与40年前事件的相关性,进而影响最终解谜。
* 时间:程序员事件和妹妹事件发生前。男女主没有确定恋爱关系。
    }
{1}{湖畔【大地图】
	* 某夜,男主路过未名湖边,发现湖心岛上有一群戴着面具的人。【BGM湖畔-神秘】
	* 仔细观察,发现其中一人似乎是女主,那个人向自己打招呼。
	* 男主走上湖心岛,那人摘下面具,果然是女主。
	* 女主向男主介绍,这些是文学社的同学,正在举行“假面诗会”。
	* 男主尚未搞懂什么是假面诗会,就被旁边的人打发走了。}
{2}{社团活动室。
	* 男主和女主聊起昨天的事情。男主了解到,女主是文学社的社员。
	* “假面诗会”是文学的例行活动:深夜在校园选取一个地方,每个人带着面具和自己的诗作,不署名,让大家找出各首诗的作者。【如果感觉没必要写文学社,也可以设定成“中文系的活动”】
	* 男主突然头脑发热,问女主自己能否参加(追求阶段,创造共同经历)/女主邀请男主参加。
	* 最终,男主决定参与下一次假面诗会。【可设置分歧点】}
{3}{中文系图书馆。	
* 男主从未有过创作经验,为自己一时冲动感到后悔,又很担心自己写不好在女主面前出糗。因此来到图书馆寻找参考资料。(需要吐槽,为什么校图书馆进不去)
	* 男主发现了一本诗集《博雅塔》,出版时间是40年前,集录了当年北大学生的现代诗作品。
	* 翻阅时,男主发现某首诗中的一句正是女主的微信签名,这首诗的作者名为……。男主对这首诗产生了很大的兴趣。
	* 一位老者来到书架旁,似乎是在寻找此书。男主走上前去,对方看到了《博雅塔》的封面,露出了惊喜的表情。
	* 老者打算借阅《博雅塔》,男主拍了几张照片后将书递给了老者。办理借阅手续时,男主发现这位老者正是那首诗的作者。在网上查询,发现此人是中文系的X教授,研究现代诗歌。\Ham{男主应该认识教授,民俗学老师}
	* 男主鼓起勇气,向其询问有关那首诗的事情。老者称这是自己学生时代写的,并将男主带到了自己的办公室。【可以和妹妹事件中的《湖》产生联系,比如《博雅塔》是受一本奇妙的书启发,又像诗又像散文又像小说】
	* 交谈之中,教授询问男主的名字,称其像自己的一位故人。【男主祖父】就在此时,有人上门拜访,男主前去开门,发现竟是女主。
	* 两人都倍感惊讶。女主告诉男主,X教授是自己的本科生导师。男主随即离开了。\Ham{这段可能要改,男主对女主、教授关系不确定}
	}
{4}{艰难的创作过程。男主从女主处得知了下次假面诗会的主题——《人工智能之诗》【为程序员事件埋下伏笔】。在所有人的诗中会混入一首人工智能的诗,需要挑选出来。同时,其他人同意男主参与诗会。妄想(表白诗……)}
{5}{假面之夜。	
* 诗会在静园草坪举行。一开始,主持者介绍此次新增了两人。新来的人需要提交一首署名诗,让大家了解其风格。
	* 诗会开始不久,几首风格明显的诗就被指认了作者。还剩下四首。(当大家统一意见,某首诗属于某个人时,那个人就需要回答是或否。若指认成功,那个人就需要摘下假面。)
	* 寻找女主的诗。激烈的心理活动,最终选择枝。【选对了加好感之类的】
	* 剩余三首,有一首是另一位新增者的。那个人被指认后,缓缓摘下面具。男主发现那竟然是室友。室友坏笑着说,他听到男女主的交谈,感觉有意思就也来参加了。
	* 剩余两首,区分男主和人工智能的诗。最终女主成功指认,某首诗具有X教授的风格,应该是男主的。
	* 众人就”人工智能的诗究竟能否算作诗“展开了激烈的讨论,但后来忽然发现静园草坪上的天空十分美丽,大家都摘下面具,躺在草地上仰望星空。
	* 【若男主指认女主诗成功,解锁谈情说爱环节】女主悄悄对男主说,她指认男主的诗并不只是因为X教授的风格,其实是因为“人类的温暖”。诗言情,技法不是最重要的,她为认识了一个纯真而浪漫的男主感到欣喜。}

    }[\PackageError{tree}{Undefined option to tree: #1}{}]%
}