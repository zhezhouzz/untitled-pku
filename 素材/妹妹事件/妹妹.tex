\newcommand{\meimei}[1]{
    \IfEqCase{#1}{
        {0}{
        \begin{enumerate}
            \item 妹妹在社团会议上提到自己在寻找一本书,希望大家帮忙。但她对书名讳莫如深,也不透露自己找书的意图。此外还说,这本书在全世界只有一册,藏在北大图书馆。
            \item 男主记得图书馆将会在校庆日附近开放,希望妹妹再等等。女主则欲言又止。
	        \item 几人离开活动室去食堂吃饭,途经图书馆,发现工地门口的告示:由于施工进度,图书馆开放再次延期。
	        \item 女主讲述了最近在中文系盛传的流言:图书馆早就装修好了,是因为闹鬼才延期开放的。
        \end{enumerate}
        }
        {1}{
        \begin{enumerate}
            	\item 男主回到宿舍,发现室友迟迟未归。躺在床上时突然收到妹妹的短信:“会长约我出来了,说可以带我进图书馆找书,哥哥你太菜了。”
	\item 惊恐而愤怒的男主从床上一跃而起,冲向图书馆。男主隐约看到妹妹站在室友旁边,正打算冲上去理论,突然发现女主也在,才稍微松了口气。妹妹对男主做了个鬼脸。
	\item 几人采用最原始的方式——翻墙,潜入了图书馆。庭院中月色如水,早已翻修一新。室友根据地上工人的鞋印,找到一处未上锁的小门。
	\item 进入图书馆,女主惊叹于新馆的结构,妹妹则直奔二楼。男主问妹妹是否可以透露书名,大家一起找。妹妹支支吾吾地回答这是一本文学类的书。男主感到意外,因为本以为妹妹要找的是写论文用的法律专著。
	\item 女主偷偷打听到了书的名字,并查到这本书存放在地下室,几人开始琢磨如何前往地下室。
	\item 妹妹突然说自己听到了声音,并独自奔向一间阅览室(歌剧魅影)。其他人赶快追了上去,但此时妹妹已不见踪影。
	\item 男主、女主和会长在图书馆中到处寻找妹妹,大声呼喊,但无人回应。寻找过程中,发现了暗门、被堵死的楼道、上锁的门等机关,但仍没有找到妹妹。
	\item 大约半小时后,男主透过百讲的玻璃,看到巨大的黑影。(歌剧魅影)
	\item 男主来到百讲前,发现妹妹神情恍惚但笑容满面,手中拿着一本破旧的书。妹妹说,“北大的地下空间都是联通的”,刚才她从图书馆的地下走到了百讲,还找到了自己想要的书。
        \end{enumerate}
        }
        {2}{
        \begin{enumerate}
            	\item 前一晚图书馆探险过后,妹妹的状态不太对劲(表现得很霸气,又很傻气),男主决定偷偷跟着妹妹。【大地图模式】
	\item 一日之间,妹妹走进了各院系的图书馆。(中文、历史、社会学……)男主与管理员、同学对话,了解妹妹的动向。其中加入和程序员的互动。
	\item 最终,妹妹停在了一间办公室门前。男主发现这正是X教授的办公室。
	\item 待妹妹离开后,男主进入了办公室,向X教授说明“妹妹有点不对劲”,并打听那本书的内容。此时女主恰好来访。
	\item X教授说,妹妹所找的书名为《湖》,是一本“元小说”。以北大为背景,讲述了一个少女寻找一本名为《湖》的诗集的故事。【如果X教授支线达成,会得知此事与40年前的校庆有关,据传是X教授的同学编写的】。 
        \end{enumerate}
        }
        {3}{
            \begin{enumerate}
        	\item 晚上,男主对妹妹仍然不放心。和女主告别后,决定潜伏在图书馆旁边。\ZZ{为什么不放心}
	\item 深夜,妹妹果然出现,并准备翻墙。比男主领先一步冲出来的是同样在旁边等候多时的女主。
	\item 两人追问妹妹的来意,多时妹妹才开口:“昨天一位老者带妹妹进入了图书馆的地下,帮她找到了书,并委托妹妹带一些院系图书馆的书过来。”这位老者自称“图书馆的守墓人”。
	\item 此时,他们听到了老者的声音。在声音的引导下,男主一行人推开了暗门,来到了地下。
	\item 图书馆的地下并非阴暗潮湿,而是一个空寂、明亮的场所。(参考回转企鹅罐09集,中央图书馆天空之孔分室,突出一种超现实感)
	\item 一位老者(幽灵)出现在他们背后,无视了男女主,仅和妹妹对话,让她拿出今天在院系图书馆找到的书。
	\item 幽灵翻看着,情绪愈发激动,批判当今的文化界。“礼崩乐坏”“象牙塔的坍塌”
	\item 幽灵的目光转向男女主,缓缓地说:“图书馆是思想的坟墓。你们身上没有纸张的味道……只有她可以继承我的职业,成为大图书馆的守墓人。”(指妹妹)
	\item 【此处补充幽灵教授的设定】20年代的中文教授,一个有些艺术气质的老教授,有自己的一套美学观念,而且有些固执、守旧(辜鸿铭那种感觉?)。诗集《湖》的作者,但他自己忘记了这件事(书的原型是废名的《桥》)。《湖》这本书因为神力被具象化,变成了幽灵,所以这本书就怎么都找不到了。
	\item 【此处补充地下图书馆的设定】地下图书馆同样是神力的产物,并不是真实存在的。具象化之后,幽灵自认为是“图书馆的守墓人”,不停翻看身边的书,目的是寻找记忆中一本十分重要的书《湖》。对当今时代的文化表示好奇,要求妹妹拿书给他看。认为妹妹和自己气质相合,适合做“守墓人”的继承者。(因为妹妹比较读死书,“身上有纸张的气味”。女主虽然读书多,但比较灵活)
	\item 【此处补充妹的设定】平常一直在看书;缺点是读死书,一根筋不知变通;偶尔表现得强气而霸道,实际上很没底气。这些东西都是最后要成长的。
	\item 男女主答应帮幽灵教授寻找《湖》,女主又和幽灵就当今文化进行了一番嘴炮。(妹妹的矛盾和成长,待补完)
	 \end{enumerate}
        }
    }[\PackageError{tree}{Undefined option to tree: #1}{}]%
    }