\newcommand{\nvzhumeng}[1]{
    \IfEqCase{#1}{
        {1}{
            【男主角】遇到了一只可爱的猫猫。猫猫非常聪明听话,让它做什么它就做什么.渐渐地主角开始犯懒,让猫猫帮着打扫房间(可选选项),做家务,没想到猫猫都可以胜任。
        }
        {2}{
            渐渐地主角开始犯懒,让猫猫帮着打扫房间(可选选项),做家务,没想到猫猫都可以胜任。猫猫也和主角团熟悉起来,女主最喜欢猫猫了,因为它和女主过去养的一只猫很像。
        }
        {3}{
            与此同时,男主角和女主角参加【民俗学】课程,课程的大作业是探寻一个本地的民俗传说。主角选择了【传说A:据说是可以吞噬人灵魂的石像】,而女主角选择了【传说B】。男主角认为自己在专业课上比不过北大学霸,但是在民俗学这门“不务正业的课”上面总要证明自己,所以他提出两个人都先收集资料,看谁收集的资料全,那么就采用谁的主题。
        }
        {4}{
            在收集资料的途中,猫猫帮助了男主角很多,男主角了解到【传说A】实际上是起源于四十年前,那时候有一个北大宿舍里的所有学生都突然昏迷不醒。他们已经住院四十年了,同时流传出来他们的灵魂被吃掉了。这件事之后,没几年北大都会有昏迷不醒的学生出现。
        }
        {5}{
            主角去了医院,在医院找到了一个昏迷不醒的人和他们的家属,像他们了解情况之后,家属给了他一些当事人的所有物。他们似乎是前往山里探险,主角还拿到了十几年前他们拍摄的【底片】。            主角将底片洗出来,发现了那是一个山洞之中的石像,看上去不像是假的。得到了决定性证据的主角十分高兴,他在资料方面收集得更全,因而也让女主角同意最后大作业写【传说A】。
        }
        {6}{
            猫猫和女女主角一切准备大作业。女主角在进行PPT制作的同时,男主角感觉身体越来越不好,白天总是处于半梦半醒的状态,晚上则总是梦见底片里的那个山洞。
        }
        {7}{
            猫猫和女女主角一切准备大作业。女主角在进行PPT制作的同时,男主角感觉身体越来越不好,白天总是处于半梦半醒的状态,晚上则总是梦见底片里的那个山洞。
        }
        {8}{
            他开始逐渐怀疑自己是不是也被吞噬了灵魂。但是这个时候他已经卧病在床,女主角担心打电话过来,他甚至没有办法接电话。这时候猫猫跑过来看了他一眼,那个眼神让主角记忆深刻,然后主角就陷入了梦境。他一直徘徊在那个山洞之中,原来他被封印在了山洞中的石像之中。过去几十年被封印的灵魂们也都在,他们告诉主角,任何直接或者间接看到石像的人,最后都会被石像收走灵魂。这里的人等了几十年,也看不到任何被解救的希望,男主角听完后陷入了绝望。
        }
        {9}{
            没想到不久之后,猫猫竟然找到了山洞。主角大声对着猫猫疾呼,让猫猫来救自己。猫猫似乎听懂了主角的话,尝试着打碎石像。没想到每次攻击石像,石像身上都发出一道【电光】,将猫猫电得直叫。男主角处于灵魂的视角,他可以看到猫猫虽然身体受伤不大,但是灵魂上似乎变得萎靡。相对的,石像也似乎受到了一些伤害。男主角和被封印的人非常激动,他们看到了被解救的希望,因而指挥猫猫继续努力。随着猫猫变得越来越虚弱,男主角开始觉得让猫猫这样牺牲救自己不太好,有些愧疚。他想要阻止猫猫,但是周围的人却不断地要求猫猫继续努力,男主角不好开口说出阻止的话(有组织的选项,但是是灰色的选不了),毕竟这里有这么多人等到救援。再说了,女主角也看过【底片】,说不定过段时间也会被封印进来,男主角不想女主角也落到和自己一个下场。他用这个说法说服了自己。最后,猫猫终于打碎了石像,而主角也回到了自己的身体之中。
        }
        {10}{
            猫猫虽然也在不久之后回来了,但是却变得痴痴呆呆的,也不吃东西,逐渐消瘦下去。女主角虽然不知道发生了什么事,但是还是感到非常伤心,甚至表示希望想尽一切办法救助猫猫。
        }
        {11}{
            主角开始反思自己,当时他为了让猫猫救自己,甚至用女主角来当借口。主角感到十分愧疚,他想一切要是可以重来多好,他不想再坚持【传说A】了,这样一切悲剧都不会发生。
        }
    }[\PackageError{tree}{Undefined option to tree: #1}{}]%
}

\newcommand{\nvzhu}[1]{
    \IfEqCase{#1}{
        {0}{
            男主角和女主角选课:【民俗学】。
        }
        {1}{
            主角醒来,发现之前的一切似乎是一个梦,而自己又重新回到了 \DTMdate{\mydate+ 14},只是他不再是人而是那只猫猫。【原身】遇到了一只可爱的猫猫(主角)。猫猫非常聪明听话,让它做什么它就做什么.渐渐地原身开始犯懒,让猫猫帮着打扫房间(可选选项),做家务(复刻一周目选项,主角恨得牙痒痒但是都可以胜任了。
        }
        {2}{
            渐渐地原身开始犯懒,让猫猫帮着打扫房间(可选选项),做家务(复刻一周目选项,主角恨得牙痒痒但是都可以胜任了。主角到处探索,适应身体,并且偷偷开始调查传说。
        }
        {3}{
            与此同时,原身和女主角参加【民俗学】课程,课程的大作业是探寻一个本地的民俗传说。原身选择了【传说A:据说是可以吞噬人灵魂的石像】,而女主角选择了【传说B】。原身认为自己在专业课上比不过北大学霸,但是在民俗学这门“不务正业的课”上面总要证明自己,所以他提出两个人都先收集资料,看谁收集的资料全,那么就采用谁的主题。变成猫猫的主角想要阻止原身选择【传说A】,但是原身非常固执(复刻第一周目选项)。
        }
        {4}{
            在收集资料的途中,男主角了解到【传说A】实际上是起源于四十年前,那时候有一个北大宿舍里的所有学生都突然昏迷不醒。他们已经住院四十年了,同时流传出来他们的灵魂被吃掉了。这件事之后,没几年北大都会有昏迷不醒的学生出现。主角意识到,对于原身来说“放弃【传说A】”很简单,但是对于不能说话的猫猫来说却千难万难。原身最后在医院得到了【底片】,他踌躇满志,认为这次大作业肯定要选自己的题目了。
        }
        {5}{
            原身去了医院,在医院找到了一个昏迷不醒的人和他们的家属,像他们了解情况之后,家属给了他一些当事人的所有物。他们似乎是前往山里探险,原身还拿到了十几年前他们拍摄的【底片】。            原身将底片洗出来,发现了那是一个山洞之中的石像,看上去不像是假的。得到了决定性证据的原身十分高兴,他在资料方面收集得更全,他踌躇满志,认为这次大作业肯定要选自己的题目了。主角意识到,对于原身来说“放弃【传说A】”很简单,但是对于不能说话的猫猫来说却千难万难。主角猫猫偷偷将底片毁掉。
        }
        {6}{
            原身在和女主角对资料的时候,发现底片竟然已经被猫猫毁了,气得要命想要惩罚猫猫,但是被女主角拦下。失落的原身不想在大作业划水,他认为女主角收集了资料,那么就应该由自己来做PPT。女主角意识到男主角的低落,是不是来看望他。主角猫猫觉得阻止了原身作死大功告成的时候,却意外地注意到女主角离开时的眼神十分熟悉,就像是一周目猫猫的眼神一样。
        }
        {7}{
            主角猫猫越来越不安,他发现女主角神神秘秘地不知道在干什么,于是主角猫猫混在女主角身边,两个人逐渐熟悉。
        }
        {8}{
            有一天主角猫猫趁女主角外出的时候来到她的卧室,发现她竟然在整理【传说A】的资料。主角猫猫通过这些天对女主角的了解,明白了她的想法。她一定是不想看着男主角的努力白费,因此想要重新拍一张山洞里的石像的照片,最后大作业还是选择【传说A】。主角猫猫想,为什么这个女人这么傻,并且它意识到女主角可能已经动身前往山里了。女主角的桌子上标注了她认为可能的几个地点,主角猫猫根据一周目的记忆,判断哪一个最有可能(小游戏),并且它体会到一周目猫猫能找到自己有多不容易。
        }
        {9}{
            主角猫猫选择到了正确的地点,发现女主角已经倒在了石像旁边,她的灵魂正在被石像吞噬。主角猫猫选择和一周目猫猫一样,试着打碎石像,而石像每次反击的【闪电】都让他痛彻心扉。被封印的灵魂们不断地指挥主角猫猫,却不体谅它的痛苦,让主角猫猫十分不爽,但是他还是想要赶紧打碎石像,拯救女主角。这时候,女主角却勉强站了起来,帮助主角猫猫挡下了闪电。她要和主角猫猫一切承担,不想看到主角猫猫一个人牺牲。女主角宁愿猫猫补救自己,也不想看到猫猫受伤。女主角说的话正是一周目主角没法选的选项,主角猫猫因而被感动,对女主角的好感度有了一个跃升。同时被打动的还有被封印的灵魂们,他们开始努力挣扎,试着抓住【闪电】,不让【闪电】继续伤害女主角和主角猫猫。而这导致了石像像蜡像一样融化消失。原来,石像的力量来源于被封印的人的执念和怨恨,而现在这些人宁愿自己被封印也不想看到有人为他们牺牲,这从根本上消解了石像的力量。石像因而完全消失在了时空之中。
        }
        {10}{
            男主角再次醒来,发现自己正站在医院之中收集资料,上次预约他因为生病没有来。躺在病床上的人并非因为什么石像而昏迷,他是出了意外事故,有人以讹传讹。这下子【传说A】不攻自破,已经不用再收集资料了。他们最后选择使用【传说B】。
        }
        {11}{
            男主角得到了一个中规中矩的成绩。他被第二周目的女主角的自我牺牲所感动,对女主角暗生情愫,并且因为和女主角生活过一段时间,所以对她格外了解。女主角并不知道男主角变成过猫,但是通过蛛丝马迹她开始有所怀疑。
        }
    }[\PackageError{tree}{Undefined option to tree: #1}{}]%
}

\newcommand{\nvzhuer}[1]{
    \IfEqCase{#1}{
        {0}{
            在妹妹事件的最后,老人在消失之前说,他的意识曾经沉沦了很久,知道十几年前才复苏。并且他觉得主角团总是可以遇到灵异事件,这不太正常,或许冥冥之中有幕后的黑手。这是在战斗之中说的,男主角认为他是在离间大家,因而并没有在意。在战斗之后,解决完足球事件他心中的疑惑逐渐变深,因为那颗足球也是十几年前被遗忘的。  
        }
        {1}{
            在女主事件之中,男主角失去了猫猫。他依照自己作为猫的行为方式,和对于猫的记忆寻找猫猫的过去。他阴差阳错找到了女主角家附近,并且意识到自己化身的猫猫和女主角五岁时养的那只非常像。邻居说他记得很清楚,因为女主曾经差点被一个足球打中,是小猫跳起来救了小时候的女主,因而在身上留下了疤痕。但是,猫不可能活15年还是长得一样。是那只猫的后代,还是另有原因?
        }
        {2}{
            男主角记起在程序员造成的骚动之中,他制造的人工智能曾经对每个人类都加以嘲讽,除了女主角。男主角找到程序员,男主角想重新和修复之后的人工智能对话。人工智能表示,他在网络上除了学籍信息和身份信息,查不到任何关于女主角的痕迹,人工智能对她一无所知自然没法嘲讽。
        }
        {3}{
            男主的疑惑加深,女主角最近在准备一个导师交给她的重要任务。平时一直很乐观的的女主角现在变得有些焦虑,男主角因而没有找她询问。男主角舒了一口气,却又不可抑制地开始自己调查女主角,知道有一天身后有一个人走来。原来是程序员,上次主角莫名其妙的问了关于女主角的事情,程序员就记在心里,最近发现男主角频繁查资料,所以就黑了男主角的电脑,发现了他的举动。男主角被迫将这一切告诉给了程序员,程序员说他会保密,并且帮助男主角。
        }
        {4}{
            程序员的确保守了秘密,但是男主角的异常举止却瞒不住舍友。在舍友的逼问下,男主角就算承认暗恋女主也没有说出他对于女主角的怀疑。舍友放过了男主,转身找到程序员,说已经知道了而关于男主角对于女主角的事情,程序员不知道有诈,就说出了男主角对于女主的怀疑。
        }
        {5}{
            到了第二天,男主角就被骑在身上的妹妹摇醒,原来舍友转手就将这件事告诉给了妹妹,这下子除了女主角之外主角团的都知道了。无奈之下,男主角只好找到大家开了一个背着女主角的秘密会议,大家开始商量对策。在会议之中,大家提出了各种的猜测。程序员认为女主角就是幕后黑手,具有强大的灵力,设计了种种事件想要实现她不可告人的秘密。妹妹认为女主角的确有灵力,但是是出于好心,她设计的事件并未伤害到任何人。舍友开玩笑说女主是凉宫春日的类型,潜意识做这一切,都是为了吸引男主角的注意力,被害羞的男主角否决掉。大家反过来问男主角到底是怎么想的,男主角沉默,他自己也不知道。
        }
        {6}{
            随着时间的推进,男主角发现学校之中的人越来越模板化,就像是游戏之中低智能的NPC一样。而唯一保持正常的女主角和女主角认识或者记得的人。
        }
        {7}{
            主角团再次相聚,大家认为这可能是因为女主角心情不好。大家经过讨论之后,舍友的理论占了主流,需要男主角潜伏到女主角身边,旁敲侧击地搞清楚发生了什么并且将女主角哄好。
        }
        {8}{
            开完会的第二天,男主角发现主角团遗忘了昨天的事情,他们“只记得女主角知道的事情”。男主角突然觉得不寒而栗,他不觉得自己是特别的,也不知道自己可以坚持多久。女主角似乎有心事,她约了男主角去了咖啡厅,男主角问了半天女主只是说导师的任务让她觉得苦恼,男主角说要不然放置一段时间导师的任务,好好生活。问清了导师的任务的截止时间,男主角认为女主可以放三天假期。女主角说日常生活十分单调无聊已经不能让她提起兴趣了,男主角想到了变得循规蹈矩的众人,意识到关键点来了,他忍不住脱口而出,问自己可不可以做女主角的男朋友。
        }
        {9}{
            主角一边和女主角甜蜜,一边调查女主角的方方面面。他心中觉得纠结和痛苦,因为他还是欺骗了女主的感情,更让他痛苦的是他逐渐沉迷其中。
        }
        {10}{
            主角一边和女主角甜蜜,一边调查女主角的方方面面。他心中觉得纠结和痛苦,因为他还是欺骗了女主的感情,更让他痛苦的是他逐渐沉迷其中。
        }
        {11}{
            事情开始好转,学校之中的人都慢慢恢复正常,只是他们依旧没有记起女主角的事情。女主角也恢复了优等生的姿态,男主角向她询问导师的任务如何,女主角说她已经觉得没事了。事情解决,男主角却觉得哪里不对。直到他无意之中提到消失的猫猫,女主角想要做出难过的表情却显得十分僵硬,男主角认定女主角在勉强自己。女主角在进行了完美的演讲之后,并不记得自己讲了什么,她好像快进了很多超出她能力的事情。但是在男主角表现出担心的时候,女主角表示只要大家都喜欢这样的自己,又有什么不可以呢?她软弱的时候只要男主角能看到就好了。
        }
        {12}{
            男主角在一次完美的约会之后,觉得十分舒适,但是他随意地和女主角闲聊,说了一句话之后女主角突然面露异样的表情。男主角突然意识到女主角刚才和自己约会也在快进,而自己甚至没有察觉到这一点,作为男主角不合格。男主角想要道歉,却发现女主角再次变得完美,他甚至找不到和软弱的女主角道歉的机会了。
        }
        {13}{
            在恢复正常的世界里,男主角打算独自探寻真相,主角团也不支持。最后在爷爷的帮助下,男主角发现导师有问题,于是【克服了种种困难】来到了导师那里,进行【某种boss战】。
        }
        {14}{
            导师告诉男主角,这是他作为人类利用灵异之力的一个实验,人类如果可以操纵灵异之力,就像是学会使用火焰的原始人一样得到了巨大的飞跃。而女主角是他的第一个试验品,他想要将一个普通人改造成为人类之中的精英。因而他让普通的小女孩转校到北大附小(如果灵异事件只发生在北大,设定小学和中学也在北大之内),并且在他设计的种种事件之下,让女主角以普通人的资质不断地努力成长,最后考上了北大。但是导师并不满意,他想要更进一步,所以他提取了女主角的记忆设计了一些事件(如果没有男主角,那么将会是女主角和猫猫解决民俗学事件),希望女主角继续成长,但是女主角在事件之中表现的能力开始失去成长性,像是触及了资质的上限。导师决定给女主角更大的压力,开始操作整个学校,push女主角进步,可适得其反,反倒让女主角失去了上进心。但是无心插柳柳成荫,没想到男主角的告白竟然让女主角再次爆发出了上进心,在被人爱的时候女主角想要表现得更好,这让女主角突破了自己的上限——这次她无意识地利用灵异力量改造了自己,让自己成为了自己理想之中的超人。导师因而很感激男主角,解释完一切之后,导师认为自己的试验非常成功,但是男主角认为这是不对的。【阐述普通人理论】,男主角最后打破了导师的意识,将不完美的女主角找了回来,两个人的关系更进一步。
        }
    }[\PackageError{tree}{Undefined option to tree: #1}{}]%
}