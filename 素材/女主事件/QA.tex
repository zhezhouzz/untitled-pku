一个专门用来讲男女主恋爱的大事件。比如一个猫的事件。

Q:如何让玩家对男女主的爱情有代入感?如何承认官方CP?
A:换视角,一种更加高级的换视角。让玩家可以同时从男女主的角度来看待对方。最好让玩家在女主角的视角下觉得男主角很温柔可爱,必须要把男主角娶到手。

Q:一般来说女主角是不可操纵角色。如何以此为前提,在只操作男主的情况下换视角?
A:(我从起点流行的游戏化写作学到的一招)只换角色的身份,不换角色本身。首先主角和一只猫遭遇了神秘事件,并且在事件之中遇到危险,猫猫对主角不离不弃,傻傻地帮助主角同时卖萌,主角对猫猫很喜爱,要是人的话甚至想娶了她。但是主角在这个“游戏”之中只打到了normal end,时间解决的不完整,猫猫为了主角牺牲了,主角抱着猫猫痛哭。但是因为事件没解决完美,主角变成了猫猫,同时遇到了女主,并且陷入了相通的时间。有了一周目记忆的主角想要避免normal end的失败,但是是猫猫没法说话,只能一边卖萌一边想办法引导女主角,。最后在女主对于猫猫的牺牲下,(插入:女主说猫猫你救了我这么多次,这次换我救你了),达成happy end。好处是玩家可以把一周目对于猫猫的爱共情到女主对于二周目主角的爱,这样就很合理。女主角对猫猫吐露心声也更加真实。两个周目实际上是对一个故事的两次叙述,可以埋更多的伏笔,也是对一个恋爱的双方视角的阐述——主角在一周目实际上是“一般故事主角视角”,是被猫猫爱的一方,在这个周目感受猫猫(男主角)的可爱,解释为什么女主会爱男主;主角在二周目实际上是“一般故事配角视角”,扮演一只猫猫去爱一个“一般故事主角”,在这个周目感受一般故事主角(女主角)的可爱(因为女主角反过来救了配角),解释主角为什么会喜欢女主角。