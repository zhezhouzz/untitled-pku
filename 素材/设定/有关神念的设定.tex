# 神念设定
##一、有关神念
* 一种超自然力量,源于人类的愿望。
	* 故事的内核。在北大实现理想。
* 个体实现愿望的意志越强烈,神念也越强。
	* 一般来说,比较佛比较摸的人神念较少,胸怀大志的人神念较多。
	* 个体身上的神念是不断变化的。	
	* 值得说明的是,个体并不一定真正了解自己的愿望,即使这种愿望产生了大量神念。
* 神念因人类产生,附着于人类身上,可以转移至物体。
	* 当个体去世时,神念转移至其挂念的物体上。
	* 附着于物体的神念会随时间逐渐消散。
* 神念的外形是具有特殊颜色的火焰。不同人的颜色不同,明亮程度也不同。利用某些手段,可以观测到个体身上的神念。
	* 在本作中,教授利用“六尺镜”,可以观测到神念。
* 当人与人或者人与物产生联系,且两者神念的颜色相近,就会产生“神念共鸣”,引发奇迹。
	* 联系:比如好奇、关心、喜欢。kizuna dakara
	* 颜色相近:暗示两个人的愿望一致,共同努力,自然会实现理想。
	* 奇迹:超自然现象,本作的核心剧情。具体而言,并不是愿望的直接实现,而是愿望的具象化。
* 神念共鸣需要两方的神念达到一定临界值。引发奇迹后,神念会有一定的损耗。
	* 两方神念的颜色越接近,两方的联系越紧密,这一临界值越低。
	* 损耗的原因:愿望实现了,由于神念与实现愿望的意志正相关,自然会减弱。
* 原则上每个人都拥有神念,但“六尺镜”只在北大生效,“奇迹”也只在北大发生。
	* 具体理由不解释。可以是“北大神念平均值超高”“气场比较好”“未名湖是出现奇迹的必要条件”blablabla

	
##二、故事设定
* 主角团的神念色
	* 女主:“没有颜色,却无比耀眼。我不知道她的颜色,我不知道是何种愿望,让她成为了她。”
	* 男主:“我的身上……什么也没有。灰暗,空虚。镜子映出一张沮丧的脸。”——“你就是一块六尺镜。你能洞悉他们的愿望,这便成了你的光芒。”
	* 室友:激情的红色与稳重的蓝色相融合,基佬紫。
		* 也可以是黄+红,后续随意设定。
* 女主的设定
	* 拥有巨量神念。
	* 拥有无色的神念。因此,可以和所有颜色的神念共鸣。
		* 可以描述的文艺一点。“仿佛把所有的颜色凝结于一点……光芒夺目的彩虹……”
	* 并未察觉自己的愿望。自称“我没有什么特别想要的……如果非要说的话,幸福的生活下去?”
		* 一方面身负巨量神念,另一方面又说自己没什么愿望。这让教授(以及后来获得六尺镜的男主)非常吃惊,产生了兴趣,执行了后续的计划。
	* 巨量神念来自于奉献精神。“如果想得到幸福的话,首先要让身边的人幸福起来吧。”女主的愿望是“让所有人都能实现愿望。”
		* 这种奉献精神一部分来源于个人意愿,另一部分也来自于他者的期待。
	* 女主人物矛盾来自于超我与自我的冲突。“不断的奉献真的能让自己幸福吗?甚至……真的能让他人幸福吗?”她觉得自己应该成为完美的奉献者,她认为社会也是这样期待的,因此压抑自身的感受。
	* 这一矛盾在女主事件二中爆发。
		* 教授发现,所有人身上的神念都在消失,只有女主的神念不断增长。
		* 失去神念的人变得机械木讷,而吸取了他人神念的女主变得亢奋而痛苦。(不是故意吸取的)
	* 实际上,女主真正的愿望,更深层次的愿望就是“幸福地生活”。她的巨量神念,也来源于生命力,来源于对“生”的投入和热忱。
	* 最终实现自我与超我的和解,和男主共同完成了“幸福的生活”。
* 前几个事件的解释
	* 引发者均为女主,且都是不自知的。引发原因是女主对某物件感兴趣&与某人关系紧密。
	* 石舫:燕元燕火的愿望。
		* 在刚开学的时候,女主结识了元火娘姐妹,她们称“想让燕园回到古代的样子”,并要求女主帮忙。
		* 女主感觉莫名其妙,但还是把这件事放在了心上。奇迹因此发生。(但由于女主的神念不如后来多,只引发了“幻象”,而没有把全校都复古)
		* 女主把燕元燕火的事告诉了教授。教授联想到学校最近出现的灵异事件以及40年前的事件,意识到女主的能力可能与灵异有关,因此展开了后续的计划。
		* 从某种意义来讲真正的始作俑者是元火娘,动机不需要解释……(元火娘是北大的化身,最大的奇迹……)
	* 猫:40年前石像事件的受害者,渴望洞悉真相的神念附着在了物品A上。教授将A交给女主,女主引发奇迹,奇迹的内容是人变猫。
		* 需要补充一个物件A。作为40年前事件的证物之一,由教授交给女主。
		* 总之要补充一个线索,论证女主-教授-灵异的相关性。
	* 程序员:程序员的愿望是,创造出理想自我“高性能的机器,回绝一切激情”。
	* 妹妹:幽灵教授的神念附着在《湖》上,X教授向女主介绍这本书,进而引发奇迹(书变成幽灵)。
	* 室友:室友的愿望是xxx,待补完。
* 教授与女主的关系
	* 教授研究民俗学,在2010年左右取得突破,获得了六尺镜。(从博雅塔下挖出来的x)
	* 经过几年的研究,教授大致掌握了神念的性质。
	* 女主入学。透过六尺镜,教授惊于此人身负的庞大神念。同时女主的外在表现与其神念的强度也不相符,出现核心悬念“女主的愿望是什么?”
		* 男主在获得六尺镜之后,也产生了相似的疑问。
	* 由于石舫事件,教授发现女主的另一个特点——无色神念,便于引发奇迹。
	* 教授利用六尺镜,搜集了校园中附有大量神念的物品(比如《湖》),总结出一份报告交给女主,称其为“40年前的调查报告”。目的是让女主去引发奇迹。
	* 教授仅提供一种引发、驱动。并不是立场不合的boss。
	* 女主的核心矛盾是感觉自己工具化了,失去了自我。教授在这个过程中确实起到了负面影响,后面可以反思。

##三、时间线
* 主角团逐渐怀疑女主。
	* 需要补充一些证据,在前面埋好伏笔。
* 男主发现身边的人变得逐渐机械,行将枯萎的世界。
	* 男主不受影响,因为男主是镜子,自己不拥有神念。
* 男主尝试接近女主。趁机表白。
* 某个偶然的契机,男主获得了六尺镜。
	* 教授故意交给男主的。因为教授也发现了男主不受影响,希望他去拯救女主。
* 继续获得线索,比如教授桌上有一份文件/女主最初持有的文件,里面早就写好了后续灵异事件的发生地点。
	* 很多线索是教授故意给的。
* 矛盾转向教授。文件中预告了最后的事件,在决战地(比如博雅塔),男主与教授相见。
* 教授开始向男主解释神念,并讲述了女主的特殊性,要求男主去拯救女主。
* 和女主互动。终结事件,情感升华。