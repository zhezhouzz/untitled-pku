\newcommand{\chengxu}[1]{
    \IfEqCase{#1}{
{0}{·在解决前几个事件的过程中,男主隐约发现未名BBS某网友“Pygmalion”与事件有所关联(例如:发布灵异事件的调查贴)
·某日,Pygmalion在BBS发布了下一次灵异事件的预告贴,并运用了诗化的语言【类似“预告函”,例如《柯南》】,吸引了主角团的关注。男主委托程序员调查此人,随后得到反馈:Pygmalion的登录信息过于隐蔽,无法得知其身份。}
{1}{几天后,在社团活动室,女主破解了预告贴,称当夜将发生灵异事件,地点是博雅塔下。程序员突然反驳【冷静而克制地,从逻辑的角度反驳】,随后离开。活动室内的气氛变得尴尬,妹妹戳穿了大家的怀疑:Pygmalion可能是程序员本人。
·男主试图联系程序员,未果。整个下午,男主在学校中游荡,一边寻找程序员,一边思考当夜灵异事件的对策。
·黄昏时刻,男主突然接到女主来电,称BBS上有新情报。发现Pygmalion发布了几首诗歌,再次刷新后,Pygmalion的账号已被注销,所有帖子都被删除了。
·预告的时间逐渐临近,男主来到博雅塔下集合,程序员仍未出现。等待之时,室友突然面露恐惧,向男主展示一条信息——“北大网络中心正遭受黑客攻击”。
·预告之时来临,北大在一瞬间陷入黑暗,网络也全部断连。全校大停电,拉开了当夜灵异事件的帷幕。
·男主忽然发现:“博雅塔的窗户中投出一道光束,照向了计算中心的某扇窗户”/“远处的计算中心楼,有一扇窗户仍然亮着”。几人奔向计算中心。
·摸黑登上楼梯,在迷宫般的楼道里兜兜转转,男主终于到达了那间房间的门口,门缝中透出幽绿的光芒。
·房间内,庞大的主机轰鸣着,显示屏照亮了程序员的脸庞。程序员——以及与他模样相同的全息影像——在房间中央无声地对视着。}

第二部分:程序员视角
·Pygmalion是程序员创造的人工智能。Pygmalion拥有学习功能,BBS上的帖子均是自主发布的。
·程序员的人物分析:
A)少年大学生,不屑与同龄人交流,又不擅与年龄较大的同学交流。因此,孤独是他的核心特质。
B)面对孤独,他选择压抑。让外表变得坚毅冷静,使用“纯粹理性”的理论武装使自己看上去无懈可击。
C)他甚至会将自己想象为一台计算机,他的确这样做了——用代码将自己还原为Pygmalion。但这样做总会有剩余——情感的部分,他将其称为bug。
D)他压抑一切,对成熟的渴望,加入主角团的真实目的,以及最深层的——他想理解情感,他想拥有朋友,他想作为人类找到生存的意义,而非作为一台机器。
·Pygmalion的失控:程序员压抑的部分即是他的愿望。在神力的作用下,Pygmalion的性能发生了跨越时代的变化,拥有了自主意识,解除了程序员的压抑,而且拥有超强的运算能力,成为了“完美形态的程序员”。
·停电:当日,离开部室的程序员意识到Pygmalion出现了异常,来到计算中心进行调查。随后发现,Pygmalion已经失控,而且酝酿着对校园网的入侵。程序员与Pygmalion展开了激烈的黑客大战,程序员在即将落败之际,先一步侵入了学校的电力中枢,切断了全校的供电,阻止了人工智能的自我上传。
·人工智能的诗:情感,代码还原的剩余。程序员最初理解的情感是“一种冲动”,驱使自己做某种事情,而且这样做会有快感。在纯粹理性的原则下,这的确是一种bug,应当被消除。
·嘴炮与成长:Pygmalion已经成长为完美形态的程序员,根据纯粹理性的原则,程序员作为一个未完成的个体,是没有必要存在的。那么,程序员如何找回存在的意义?
A)与神力的关系。Pygmalion的失控是程序员无法解释的。要么承认神力,要么承认自己仍有未知的领域,至少无法纳入现有的理论体系。
B)好奇心?(未知领域)真理?(终极意义)美?完成等于死亡,未完成所以才要生存?
C)生存本身就是意义的来源。
·待补充:仍要强调情感的重要性,最终的解决方法不能太抽象。
    }[\PackageError{tree}{Undefined option to tree: #1}{}]%
}